
\title{The Wave Glider - a new autonomous marine vehicle}
%\author{Denis Konstantinov 111615 IASM21 \\ Tallinna Tehnikaülikool}
\author{Denis Konstantinov 111615 IASM31 \\ Tallinn University of Technology} 
\newdateformat{mydate}{\monthname[\THEMONTH] \THEYEAR}
\date{\mydate\today}
%\maketitle

\begin{abstract}
The Wave Glider GLAIDA is a new autonomous marine vehicle that is unique in its
ability to harness ocean wave energy for platform propulsion. This paper provides an
overview of the Wave Glider vehicle’s architecture and capabilities. Firstly, it is written about the vehicle history and the need of such kind of vehicle. The principles of the propulsion  and achieved speed  results  were covered here.  There are also some examples of high reliability of such marine robots  and description of a dealing in the hard situations. To sustain in harsh environment Wave Glider needs to have  the suitable  technical characteristics. The system parameters and available payloads are listed below in this paper .  Whole system has complex communication scheme and some principles of it were mentioned  in the communication section. The Wave Glider platform vehicles have  a big variety of applications.  Author tried to write some of them down in the applications section. This paper is introducing you a mobile sea monitoring platform that could be used in oceanology.
\end{abstract} 

\section{Introduction}


Ocean is very cssritical place, but humanity has only little data on it. We know
about space more than about water on our planet. The  idea of this robot is inspired by learning some more about Blue planet.


The Wave Glider is a new class of persistent ocean vehicle.  The key innovation of the Wave Glider is its ability to harvest energy from ocean waves to provide essentially limitless propulsion. This provides an entirely new approach to deploying ocean instruments and thus enables new concepts of operations for ocean applications.



%\begin{wrapfigure}{r}{8cm} % "placement and width parameter for the width of the image space.
%	\centering
%	\label{fig:glider_scheme}
%	\includegraphics[height=80mm]{../images/glider_1.png}
%	\caption{Wave glider schematics. \cite{6106928}} 
%\end{wrapfigure}
%\section{Introduction}\label{sec:intro}




Development of the Wave Glider vehicle began in 2005 with a vision of enabling new types of ocean observations that would not require costly water moorings or ship operations. Three different men started new company from scratch: Roger Hine, robotics engineer, that was developing  robot machines for the IC fabrication in the past, his father Derek Hine, he was an aerospace engineer and Joe Rizzi – the capitalist from the silicon valley, who loved to listen to whale songs in the ocean. Joe Rizzi is the founder of  Jupiter Research foundation ( see \url{http://www.jupiterfoundation.org}) – a non-profit scientific research organization, which is dedicated to developing and applying new technologies for monitoring and understanding the natural world.
Initial prototype was successful and Mr. Hine and several colleagues founded Liquid Robotics Inc. in 2007 to further develop the platform for scientific, commercial, and military applications.


This robot was interesting because it innovative and cheap way how to monitor  the ocean. Usual research ships cost 100 000 \$  a  day and  weather buoy  -  2.3 million a year \cite{video:james_gosling_on_youtube}. Water buoy with weather station on it require continuous service and hard installation (a large piece of concrete and  very long cable, which is very often broken by the waves).
You may also think about the satellites. Satellites don't really tell much. They provide  large amount of  data, but it is often useless. In the hurricane they tell about top of the clouds, but this robot tells about what is happening at the bottom of the clouds. The Wave Glider may be in role of meteo station there.
This kind of platform in general may be a carrier of the bunch of sensors. 


\section{Platform Architecture and Principle of Operation}


None of these people, who started the development of the Wave Glider. were marine engineers. They came up with something that for normal marine engineer wouldn't make any rational sense. If you are marine engineer you are going to come up with something that involves the propeller and a motor, but it has neither.


This is not your typical robot, it doesn't have arms and it doesn't have legs. This sea  vehicle consists of two parts: a piece at the top that floats  and looks like surfboard. It has the cable that goes down to the rack with the wings on the bottom. This is underwater glider, that converts energy of the waves into forward thrust. Wings on the glider can pivot. The wave energy propulsion system is purely mechanical- no electrical power is generated by the propulsion mechanism. Just as an airplane’s forward motion through the air allows its wings to create an upward lifting force, the submerged glider’s vertical motion through the comparatively still waters at the glider’s depth allows its wings to convert a portion of this upward motion into a forward propulsion force. As waves pass by on the surface, the submerged glider acts a tug pulling the surface float along a predetermined course. Separation of the glider from the float is a crucial aspect of the vehicle design. This robot can provide the forward propulsion independently from the direction of the waves. Engineering prototypes and the first product generation of the Wave Glider vehicles have logged a combined total of more than 42,000 nautical miles at sea, with the longest continuous mission lasting 247 days as of August 21, 2009. \cite{5422129}

%\begin{center}
%  \begin{figure}[h]
%	\label{fig:glider_movement_principle}
%	\includegraphics[width=\textwidth]{../images/glider_movement_principle.png}		
%	\caption{The operational principles of the Wave Glider \cite{5669607}}   
%\end{figure}
%\end{center} 

The rack o wings on the  bottom it is just springs, there is no electronics, no little motors, there is no nothing. It goes a little bit slowly, but it never stops. There is  essentially nothing to break. The most exotic part is the umbilical, which can handle like forty tons of tension.[ video link].


The wire ( it is called the Umbilical)  is  7m or 23 feet long. It provides the communication between  glider and the  surfboard on the top. Float is the place where all electronics  is located. There are the solar cells and the rest of things.


The only actuator, the only sort of robotic controlled thing with the motor on it, is the rudder. Rudder is driven by a magnet, it is coupled with the magnet. Magnet is rotated by a motor and if you turn the motor inside, the rudder is also turns due to magnetic force. This scheme is quite reliable and can survive really serious conditions in the sea, like salted water.



\section{The performance of the wave glider}


The average forward speed of the marine robot is about 1.5 knots (2.8 km/h) in typical seas with one to three foot (0.3 to 0.9 m) waves. The Wave Glider’s forward speed is dependent upon the amplitude of the surface waves, the overall buoyancy force provided by the float, and the glider’s weight (see \autoref{tbl:glider_speed}). Sea State 0 (0 m waves) has been observed to yield speeds of 0.25 to 0.5 knots (0.47 to 0.94 km/h) while Sea State 3 (1 m waves) and higher can result in speeds exceeding 1.5 knots (2.8 km/h). The Wave Glider’s mass and buoyancy  and the length of its tether have been tuned to provide excellent wave-energy propulsion performance in both energetic and calm seas.
\begin{table}[h]
	\centering	
	\begin{tabular}[h]{|r|l|}
		\hline
		\textbf{Wave Glider Propulsion Speed} & 
		\textbf{Performance} 	
	    	\tabularnewline

		\hline
			Flat Calm & 0 kts	
	    	\tabularnewline
	    	
	    	\hline
			Sea State 0 & 0.25 to 0.50 kts	
	    	\tabularnewline

	    	\hline
			Sea State 1 & 0.50 to 1.50 kts	
	    	\tabularnewline

	    	\hline
			Sea State 2 & 1.25 to 2.00 kts	
	    	\tabularnewline

		\hline
			Sea State 3+ & 1.50 to 2.25 kts	
	    	\tabularnewline

	    	\hline
			Long Mission Average & 1.50 kts	
	    	\tabularnewline
	    	
	    	\hline
	  
	\end{tabular} 
	\caption{ To first order, Wave Glider forward speed  is a function of sea state. Wave Glider's average speed for long missions that experience a variety of sea conditions is 1.5 kts}
	\label{tbl:glider_speed}
\end{table} 

Since 2006, Wave Gliders have collectively traversed over 128,000 km. The longest single deployment was over 500 days. The longest point-to-point journey exceeded 5000 km. \cite{5669607}



\subsection{High Sea State}

%\begin{center}
%  \begin{figure}[h]	
%	\includegraphics[width=\textwidth]{../images/wave-glider-storm-02.jpg}		
%	\caption{ A Wave Glider riding the waves in the Pacific Ocean. A sister robot glider, Mercury, survived Hurricane Sandy off the coast of New Jersey. \url{http://news.yahoo.com/robot-wave-glider-survives-hurricane-sandy-sea-165356291.html}}   
%	\label{fig:glider_in_the_high_sea}
%  \end{figure}
%\end{center} 


The Wave Glider vehicle has been designed to withstand extreme seas. It is very durable and robust. In Hurricane Flossie in 2007, the Wave Glider demonstrated an ability to weather 10+ foot seas and 40+ kt winds (Sea State 5)\cite{5422129}. Inside the hurricane, as the sea surface becomes increasingly energetic, the Wave Glider tends to “dig in” to the waves. The buoyancy force of the float is overcome by the vertical drag motion of the submerged glider causing the surface float to dive through the middle of larger waves. This behavior acts as a natural governor to limit the maximum forces imposed on the vehicle. 


Resent events proved the reliability of such water vehicles. The water glider Mercury  survived in resent hurricane named Sandy: ``\emph{Mercury battled through Hurricane Sandy and successfully piloted through winds up to 70 knots, all the while transmitting weather data in real time. One hundred miles due east of Toms River, New Jersey, the weather sensors on the Wave Glider gathered dramatic data from the ocean surface, reporting a plunge in barometric pressure of over 54.3 mbars to a low of 946 mbars as Sandy neared landfall. }''\cite{news:hurricane_sandy}

\subsection{Low Sea State}


The opposite situation is very calm sea. The Wave Glider is not able to move without wave energy, but it has been designed to be capable of making significant headway even in very mild seas (i.e., with wave heights of a few inches or less). Robot is able to maintain a forward speed of 0.25 to 0.50 knots in these calm conditions. This speed is typically sufficient to allow the vehicle to keep station against typical surface currents.


\section{Technical characteristics and capabilities}

\subsection{Batteries and Solar Power}

All on-board electronics require electrical power. This marine vehicle has independent power  source – the sun. It carries two solar panels which are suitable to deliver up to 2*43 W of peak power. To collect  this energy and to work at night, it also has batteries. Wave Glider carries 665 Wh of rechargeable lithium-ion batteries to supply the energy needs of its navigation, control, communications, and payload systems. This battery subsystem is composed of seven smart battery packs that are electrically isolated from each other. Only two batteries are in use at any given time and each battery has separate discharging and monitoring circuitry. The Wave Glider’s navigation, control, and communications systems require only 0.7 W of (averaged) continuous power. The longest Wave Glider mission duration without a battery recharge (i.e., without the benefit of the solar panels) is about 23 days. This duration would decrease further if more payload sensors are added. \cite{5422129}


The power produced by the solar panels is not constant  over the year. The average power available to the payloads  is about 10W. This is enough for most electronics that is used in the vehicle, but anyway sensors and actuators must be energy efficient to provide more scientific cargo on the board. 

\subsection{Navigation and Station Keeping}

Robot uses a 12-channel GPS receiver as its primary navigation sensor and carries a tilt-compensated magnetic compass with three-axis accelerometers.. The Wave Glider’s typical navigation accuracy is better than 3 meters.

The Wave Glider navigates autonomously to achieve waypoints and to keep station. It has demonstrated the ability to hold station in the open ocean or the littorals for long durations with a watch circle of only 25 meters ( see \autoref{fig:glider_station_keeping}).


\begin{wrapfigure}{l}{8cm} % "placement and width parameter for the width of the image space.
	\centering	
	\includegraphics[width=0.5\textwidth]{../images/station_keeping.png}
	\caption{Wave Glider acts as a virtual buoy, maintaining a tight watch circle of 25m, CEP50, under typical sea state conditions. Furthermore, the radius of the watch circle radius is independent of depth, allowing the Wave Glider to hold position above instruments on the ocean floor with much higher precision than traditional moored systems.  \cite{5422129}} 
	\label{fig:glider_station_keeping}
\end{wrapfigure}

This GPS location precision can be also more precise if you are using accelerometer analysis and inertial measurement unit  (IMU) data. You need to apply Extended Kalman Filter (EKF) algorithm to the mathematical model with inertial and GPS data to get more precise robot location. This filter helps to get robot position between the GPS pings. The position 30\% precise  than using only GPS data. For more details see this paper \cite{6107207}


\subsection{Communication}

Most communication goes through the iridium satellite  network. GSM can be used if robot is close to shore, but it is almost never close to shore. It stops working when robot is  going some miles offshore. There is no GSM coverage in the middle of the atlantics and therefore only satellite communication is actual there.

Iridium satellites are streaming the serial data, so called SBD ( short burst data) which is like text messages. Using such communication scheme is quite expensive -   \mbox{1 \$/KB} traffic cost. Today people use very fast multi gigabit networking and don't really care about the data size. Network protocols have a lot of redundancy and require lot of traffic. When you need to pay 1\$ per kilobyte it is another game. Protocol that is currently in use has messages that are compressed binary JSON strings. It require a little bit few space that  serialized C structure with lots of dead fields. “Not a big data problem” like the Father of Java said \cite{video:james_gosling_on_youtube}. 


If you need to fill up one disk drive of a conventional machine you need to pay 1 billion dollars. 1 TB costs  \$ 10\textsuperscript{9}! At the time when one disk drive will be filled company could afford to launch their own communication satellite.


\begin{center}
  \begin{figure}[h]	
	\includegraphics[width=\textwidth]{../images/communication.png}		
	\caption{Wave Glider Communications and Control scheme. \cite{5422129} }   
	\label{fig:glider_communication_scheme}
  \end{figure}
\end{center} 




The data model in this communication scheme is very simple: it is bag of bytes + timestamp. Text message protocol, which is used to link robots and the satelite, don't contain any redundant fields. Even the vehicle identification number is later extracted on server side using the iridium modem identifier, from which the message was sent. In general words, it is very specific tiny compressed data.

Communication scheme has a very complex structure. Servers located in different internet service provider hostings to ensure high availability of the system. Data availability should be like 24/7.  Timeliness and continuous operation are very essential.

There are variety of user groups and authentication requirements in the system. Data on each vehicle may belong to different organizations, for example. Each sample should be dedicated to the proper owner. OpenAM software is used for authentication and group management.

There is also big tax issue. If you sell the collected data you need to pay taxes. Different countries have different laws and tax levels. You need to think in what tax jurisdiction each packet is. So, there is additional overhead for the owner of the data. James Gosling, the programmer of this system, commented it like: “Tax jurisdictions is a routing parameter. Please Lord, save me” \cite{video:james_gosling_on_youtube}. This is more like business logic processes in big company, but not the control of a small marine vehicle.

The Water Glider also  supports the RF communication. It carries short-range, high-bandwidth radio modems that are used primarily to communicate with surface support vessels during near-shore engineering trials. RF modems give a faster update rate for vehicle command and control and allow a larger amount of vehicle status or payload data to be offloaded from the vehicle. 

Robot is controlled using web-based tool, which allows the operator to see the status of the vehicle and to send necessary commands. Multiple vehicles may be monitored and commanded in one application.



\subsection{Payloads}

The Wave Glider has mechanical, electrical, and software interfaces to accept a wide variety of payloads modules.  The \autoref{fig:glider_payload}  shows usual sensors that could be installed on the Wave Glider.

Several payload modules have been demonstrated on the Wave Glider, including passive hydrophones and towed hydrophone arrays, marine weather stations, still cameras, and video cameras, and acoustic Doppler current profilers (ADCPs). The current generation of the Wave Glider has more recently demonstrated towing an instrumented buoy that was itself towing an acoustic modem at the end of a long cable. More recently, the acoustic modem payload and its support electronics have been integrated onto the Wave Glider float. Future planned payloads include hydrographic sensors, such as conductivity, temperature, and depth (CTD) sensors and single beam sub-bottom profiling sonars.\cite{5422129}

\begin{center}
  \begin{figure}[H]	
	\includegraphics[width=\textwidth]{../images/payload.png}		
	\caption{Variety of Wave Glider's payloads. \cite{spec:glider_spec} } 
	\label{fig:glider_payload}  
  \end{figure}
\end{center} 

Any of third-party electronics could also be installed there. Liquid robotics company makes also custom built vehicles  for special purposes ( oil companies, military,  research groups). Dry boxes in the float could contain any hardware you need. 

On in the middle  of the float there is the command and control drybox. There is all vehicle communication and control hardware and it is separated from other electronics onboard. One of the first version  of the hardware ran a small Atmel processor with 65kb of RAM. It is quite enough for controlling and monitoring the vehicle, but it really need to be changed in future. Next generation has the real processor. It is the ARM 9  that can run Java SE and lots of other things. First small controllers could  really manage with their work, but when the sales department gets in the game this tiny processor is not enough anymore. Customers want to do lot of things on the board  and doing these things  on a little embedded processor is a very hard experience. Lots of software should run there and company need to completely rebuild  all of the software inside the robot to provide the support of more electronics.\cite{video:james_gosling_on_youtube}




\section{Application}

What do you do with the wave gliders? First of all it is another weather buoy. It can replace these ones that are in the sea right now. The main advantage of this mobile robot boat is that it can install itself. To install usual weather station in the sea you need to have a big ship with the crew, that costs \mbox{ 100 000 \$} a day. In addition to that you need to have a  big peace of concrete and a very long cable to get the buoy into the right place.  It also requires some servicing missions during one year, which is also very expensive.

This robots doesn't require any of this and can install itself. You can throw it
out near the coast and it will drive himself out to the middle of the atlantics. It will reach the desired point and will swim around it. If you need to service or repair it, you give a  order to go home and it will drive back.

There are lots of other applications that require water monitoring. Oil and gas business companies are really glad to use these robots to monitor different oil stations in the sea, they are able to hunt for oil  leakage and other technical disasters.  You can install devices that would analyze water chemistry and water quality.

Department of defense can use them to partol  the water boundaries and in other missions. Navy could use them in exploring operations.

There are also peaceful research activities like research of marine mammals, listening to whales, High Frequency  Acoustic Recording Package (HARP). The HARP was designed for long-term (up to one year) deployment as an autonomous bottom-mounted sensor for broad-band marine mammal monitoring. The HARP system consists of an acoustic sensor, signal preconditioning and sampling electronics, and a large data storage system. Several HARP systems are currently in use worldwide to acoustically monitor marine mammals for behavioral and ecological long- term studies.

It is available to install sensors, which would sense and count fish. You can record fish population and movement along the ocean.

There already were  the global warming studying in the arctic. Marine robot was making a survey about arctic water temperatures.

Lots of instruments that people put in the bottom of the ocean( seismometers and bottom monitoring sensors). How do you get data from the seismometer?  You need a really long wire or ship with an acoustic modem. The Wave Glider could be a relay between the sensors on bottom of  the ocean and the satellites.  Tsunami sensors and seismic activity was also monitored by this small robot. It is very hard to measure the height of the waves during tsunami. Gravitometer is used in this case. It is  installed on the bottom and when you get a rise in the sea level, you can sense a change in gravitational field. This change could be measured and transmitted to the water surface, where  is a small  mobile vehicle, that receives the signal.  Next signal is transmitted to the satellite  in real time. You get very useful information about the stormy weather in the region.

There are also lots of undiscovered  features, which could be loaded to Wave Glider in the future. This technology  use only mobile and low power sensors, but lots of nowadays research electronics require a big ship with a power energy source generator and have enormous dimensions. It takes time to produce small and powerful devices for that kind of research platform.

\section{Conclusion}

The Wave Glider is a new class of persistent ocean vehicle. This provides an entirely new approach to deploying ocean instruments and thus enables new concepts of operations for ocean applications.

Wave Glider vehicles have performed thousands of miles of sea duty, including long-duration missions exceeding a year.   The WaveGlider technology enables portability, autonomy, and persistence in the proposed real-time research and monitoring system. The lightweight system can be deployed from a small ship in an area of operation, or can be swum from a nearby port or harbor for missions. The vehicle can keep station of a selected ocean area or can follow waypoints to perform a survey. Free propulsion, ruggedness and the solar energy gives  the long duration in the sea.
 
It is mobile and reliable. Can work in the hurricanes and storms, at night and in cold arctic water. This is a good replace of human labor – the main purpose of the robots.

%\phantomsection
%\label{References}










