\clearpage\vspace*{\fill}
\section*{Annotatsioon}
Käesolevas töös kirjeldatakse kuidas rakendada teenusorienteritud klient server arhitektuuri sardmikrokontrolleril.
See on üldotstarbeline transport ja riistvara sõltumatu rakendusserver,
mis kasutab kaugprotseduuri väljakutseid kommunikatsiooni pidamiseks.
Mikrokontrolleris jooksev programm näeb välja nagu serveri teenus,
mis pakkub kasutajale ettenähtud funktsionaalsust.

Antud töö eesmärk on teha uuring teenus orienteeritud arhitektuuride ja tehnoloogiate kohta 
ning analüsida nende kasutamist piiratud ressursidega sardsüsteemide realiseerimises.

All on toodud realiseeritud teenusorienteritud sardsüsteemi kirjeldus ja sellele vastav kliendi rakendus.

Serveri pool on tehtud STM32F1 pere ARM Cortex-M3 mikrokontrolleri baasil, kus jookseb FreeRTOS reaalaja operatsioonisüsteem.
Kliendi rakendus on mobiilsel platformil kasutatav programm, mis kasutab Javas kirjutatud teegi kaugprotseduuride väljakutsemiseks.

Klindi ja serveri vaheline kommunikatsioon toimub labi traadita Bluetooth kanali kasutades JSON-RPC protokolli.
Süsteemis on tehtud mõned funktsioonid, selleks et näidata antud klient-serveri arhitektuurilisi omadusi.


\vspace{\fill}
\clearpage