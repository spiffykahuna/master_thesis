\clearpage\vspace*{\fill}
\section*{Annotatsioon}
Käesolevas töös kirjeldatakse kuidas rakendada teenusorienteritud klient server arhitektuuri sardmikrokontrolleril.
See on üldotstarbeline transport ja riistvara sõltumatu rakendusserver,
mis kasutab kaugprotseduuri väljakutseid kommunikatsiooni pidamiseks.
Mikrokontrolleris jooksev programm näeb välja nagu serveri teenus,
mis pakkub kasutajale ettenähtud funktsionaalsust.

Antud töö eesmärk on teha uuring teenus orienteeritud arhitektuuride ja tehnoloogiate kohta 
ning analüsida nende kasutamist piiratud ressursidega sardsüsteemide realiseerimises. 
Teenus orienteeritud arhitektuur võib olla realiseeritud kasutades erinevaid tehnoloogiaid: 
Kaugprotseduuride väljakutsed (RPC),
Web Services,
REST,
Microsoft DCOM(Distributed Component Object Model),
CORBA,
Java RMI ja teised.
Töö sisaldab kirjedust kõige levinumate teenus orienteeritud tehnologiate kohta: Web Services ja REST.

Web Services on sõnumitel baseeruv tehnoloogia, mis kasutab XML(SOAP) andmestruktuure sõnumite edastamiseks. 
Nende sõnumite abil on võimalik kutsuda programmi funktsioone, mis paiknevad kaugjuhtivas süsteemis.
Samuti neid sõnumeid kasutatakse andmete edastamiseks. 
Web Services arhitektuur on erinevate WS-* standartide kogum, mis sisaldab protokolle ja spetsifikatsioone veebiteenuste haldamiseks.


REST on tarkvara disaini mudel hajutatud süsteemide projekteerimiseks. 
REST kasutab standardseid veebi protokolle andmete edastamiseks ning baseerub HTTP protokollil. 
Kuna Web Services tehnoloogia kasutab sõnumeid, REST on tehtud ressurside esindamise baasil. 
Ressurss on informatsioon suvalise objekti kohta ning ressursi representatsioon või esindus on formaat, kuidas saab objekte kirjeldada.
Igal objektil peab olema temaga seotud link ning süsteemid opereevad sellega, selleks, et saada või muuta ressurside olekuid, kasutades standartseid HTTP sõnumeid: GET, PUT, POST, DELETE.
REST võimaldab ehitada väga sõltumatu veebi teenuseid ning nende realiseerimisel (nii serveril kui ka kliendil) saab kasutada igasuguseid tarkvara lahendusi ning programmeerimiskeele.

Paljud teenus orienteeritud tehnoloogiad kasutavad kaugprotseduuride väljakutsete mõisteid ning on ehitatud nende baasil. 
Antud töö ka sisaldab sarnast realisatsiooni ning kasutab JSON-RPC kaugprotseduurikutse protokolli side pidamiseks kliendi ja serveri vahel.
JSON-RPC on veel üks võimalus, kuidas saab projekteerida teenusorienteeritud süsteemi.

All on toodud realiseeritud teenusorienteritud sardsüsteemi kirjeldus ja sellele vastav kliendi rakendus.

Serveri pool on tehtud STM32F1 pere ARM Cortex-M3 mikrokontrolleri baasil, kus jookseb FreeRTOS reaalaja operatsioonisüsteem.
Antud sardsüsteemis töötab programm, mis saab kommunikatsiooni liinist sõnumeid ning kutsub tarkvara funktsioone sõltuvalt sõnumite sisust.
Server on ühendatud kontrollitava objektiga ja juhib seda. 
Juhitav objekti näidis on traadita liidesega varustatud kohvimasin, mis on võimeline teha kohvi antud süsteemi kliendi jaoks. 

Kliendi rakendus on mobiilsel seadmel olev programm, mis kasutab Javas kirjutatud teegi kaugprotseduuride väljakutsemiseks.
See on väike Android rakendus, mis laadib serverist erinevate kohvimasina produktide kirjeldust ning saab alustada nende valmistamise.

Kliendi ja serveri vaheline kommunikatsioon toimub labi traadita Bluetooth kanali kasutades JSON-RPC protokolli.
Süsteemis on tehtud mõned funktsioonid, selleks et näidata antud klient-serveri arhitektuurilisi omadusi.

Siin töös on toodud erinevad viisid, kuidas saab organiseerida andmete edastust sardsüsteemide vahel.

\vspace{\fill}
\clearpage