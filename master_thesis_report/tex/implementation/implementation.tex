\newpage
\section{Implementation}
\label{sec:implementation}
This section will cover the implementation of embedded service server and other
parts of the system. This system is only a small part of whole service
infrastructure, it does not cover discovery, addressing, authentication and
security and other essential parts of every production service.
This prototype only includes an embedded server and a client application to
demonstrate how technologies and methods already implemented in "big computer
systems" can be adapted to "small" embedded and resource-constrained devices.
Implementing a full application stack of service technologies (for example WS-*
or \gls{DPWS}) needs a lot of human and time resources. You can read
some standards( amount of pages was already mentioned above in section
\nameref{sec:adv_and_disadv_of_ws_standards}) and count how many
human*hours it would take to implement this in embedded system environment with limited amount of resources, low-level
application programming using C language and without ready made and
off-the-shelf software tools and libraries for that kind of systems. In my opinion, one master student
is unable to create a complete server solution only by himself within reasonable
time. The scope of this work requires some more resources, a team with several
members maybe, to accomplish such task.

The solution that is possible to implement during university project like this
is the research about related field and available technologies and a simple
system prototype.
I have implemented this using collected features from literature and web
resources.

This design is not based on any specific transport protocol.
Any physical protocol, that is able to deliver characters or bytes, can be used with this application.
There were no any physical and logical protocol requirements set for this prototype solution and the technologies, that company will use in future, are not specified.
Therefore, it is designed as portable as possible and operates only with character data.
It contains an simple example of data link layer protocol, that encapsulates RPC messages. 
Whole communication is transferred over serial line.


First section below covers the general architecture of implemented service.
Next come details about embedded server, which contain the description about
hardware and software platform used, program architecture and data flow.
There is also an implementation of the client side application library (also
called client stub) below.
The last section here introduces one possible client application for
this service architecture.

\subsection{System architecture and  device connection scheme}
\label{sec:device_connection_scheme}
Devices in a system may be interconnected in various ways.
Some embedded systems \textbf{do not have} any connections at all. 
Such systems only sense or control the environment and there is no need to
send data somewhere. These are usually highly embedded devices with limited
amount of functions (alarm display controller, microwave or washing machine
controller).

Another group of devices are systems that are able to interchange
information using \textbf{proprietary} communication methods at \textbf{physical
and logical} layers. These systems can send information to another systems, but
they are using non standard protocols for data transmission.

Third group uses standard (serial line, ethernet with ip protocol)
communication techniques at the physical and transport level and some \textbf{proprietary
logical} protocol above that.

Last group can integrate with all other systems and has \textbf{full
communication} possibilities. These systems use standard application protocols
and transfer data over well known channels.

Research paper \cite{lws_milanovic.pdf} introduces three architectures , that
cover three possible ways of connectivity between devices in previously 
mentioned groups: proxy, translator and full architecture.



\begin{figure}[H]
        \centering		  
		  
		\subfloat[Proxy approach provides physical and logical conversion]{
			\label{fig:proxy_arch}
			\includegraphics[width=0.3\textwidth]{../images/implementation/proxy_arch.png}
		} 
		\subfloat[Translator approach provides logical conversion]{
			\label{fig:tranlator_arch}
			\includegraphics[width=0.3\textwidth]{../images/implementation/translator_arch.png}
		}
		\subfloat[Full approach where embedded systems directly provide services for the environment]{
			\label{fig:full_arch}
			\includegraphics[width=0.3\textwidth]{../images/implementation/full_arch.png}
		} 
		       
        \caption{Possible connection architectures for the embedded services}
        \label{fig:connection_architectures}
\end{figure}




Proxy is a device which is between the client and the embedded service. It
provides services to the client and in the same time can communicate with
embedded system using closed protocols. Proxy device stores service contracts
onboard and know all specifications of connected embedded systems. 

Translator approach is similar to the Proxy, but it covers devices with proprietary
logical communication. The underlying physical transport is common to client and
service provider. The main purpose of Translator is to convert messages that
come from clients into into a logical format that the embedded system
can understand. Service contracts are stored inside services on the other
embedded systems, not on Translator. Translator may not be a separate device and
it can be only a software module.

In the Full architecture client and service provider can directly connect to
each other without need of any device in the middle. 

Our coffee machine system use closed proprietary protocol inside, but all
communication messages are transferred over standard serial line. This is more
similar to Translator approach, but there is one problem in implementing such
architecture. We cannot directly store service contract inside coffee machine
system. Coffee machine internal architecture and implementation does not allow
us to store any additional code for implementing service functionality. Internal
processor is utilized enough and there are no resources for anything else except
controlling coffee machine. In addition, the company did not
provided to me a specification of internal communication mechanism . There
are several microcontrollers inside that are controlling different machine
parts.
They provided  only the external interface communication protocol to me,
therefore my implementation is more similar to Proxy approach, where contracts are stored inside Proxy machine
and Proxy is a separate physical device.
\autoref{fig:general_system_arch} shows the general architecture of created
system.

\begin{center}
 \begin{figure}[h]
	\includegraphics[width=\textwidth]{../images/implementation/system_arch.png}
	\caption{General system architecture }
	\label{fig:general_system_arch}
 \end{figure}
\end{center}

This is a traditional client-server approach where clients ( mobile or desktop)
send the requests to the server( Proxy embedded device) over some network and
physical transport( Bluetooth, various radio frequency connections, ethernet,
USB, serial line, \ldots). Proxy server is connected to the controlled device 
(marked on the figure as Actuator), which is the coffee machine in this example
application.

There can be different clients and proxy can control and monitor various
devices, but connection scheme 
\(
{Client}\leftrightarrow{Transport}\leftrightarrow{Proxy}\leftrightarrow{Transport}\leftrightarrow{{Controlled~
or~monitored~device}} \)  is essential.

In this application mobile clients are connected through Bluetooth wireless. 
Each client may be connected using supported by the proxy device transport.
The proxy is connected to coffee machine by wires and serial line. 

Next section covers the internals of Proxy device.



\newpage
\subsection{Implementation of the embedded server}
Here will be STM32 server implementation.







 One solution is to use closed encrypted proprietary
protocol and be calm, but as it was mentioned earlier, it limits the possibility of integration
between other embedded systems. It this case all of your devices should support
that protocol and you choise of different hardware is limited. Proprietary
protocols are often vendor-specific, code is closed, documentation is not free
and all it works only with the proprietary devices from the manufacturer.


\begin{center}
 \begin{figure}[h]
	\includegraphics[width=\textwidth]{../images/implementation/embedded_server/SequenceDiagram.png}
	\caption{Client authentification process}
	\label{fig:embedded_server_login_auth}
 \end{figure}
\end{center}


reversably encrypted form.



\newpage
\subsection{General purpose service library implementation}
Here will be general purpose library implementation report.
%\newpage
\subsection{Implementation of Android client example applications}

The prototype of client-server application was designed during this research project.
The application program was executed on the Google Android operating system powered hardware.


Main idea of this application is the remote wireless control of some electronic device. 
The controlled device example here is a coffee machine, that has some functions like preparing a cup of coffee.
We needed that these functions became available for remote control.

The developed application is a small application with trivial user interface, which allows to prepare products by selecting a product from a small catalogue. 
The screen shot of a ready application is provided  in the \autoref{fig:android_app_screen}

\begin{center}
 \begin{figure}[h]
	\includegraphics[width=\textwidth]{../images/implementation/android_app_screen.png}
	\caption{The Android application visual interface }
	\label{fig:android_app_screen}
 \end{figure}
\end{center}

The usage scenario starts from the connection to coffee machine.
There is a small button with Bluetooth icon in the top right corner of the application.
This button activates a device select dialog, where you can find a remote device by name and MAC address and connect by selecting it in the list.
When connection is established some available products become showed in the horizontal list layout.
It is able to point to each of these products and open a dialog box with detailed information about each product.
User can start coffee preparing from that dialog.
While operation is in progress, it is still possible to cancel the product preparing operation.

I will not cover in details the whole process of the user interface and application creation.
This is out of the scope of this research work and it requires some additional domain knowledge. 
You need to get familiar with Google Android development tools,
read a documentation course, follow the tutorials and study by doing.
There are available lots of application examples and it is not very hard to produce a similar application.
This application is partially based on the Bluetooth chat communication example from the Android Software development kit.

Whole communication is performed over wireless Bluetooth protocol.
It is assumed, that Bluetooth is the abstract transport that can send data bytes over radio link.
The Android Bluetooth API functions can search to nearby Bluetooth devices and provide you a list of \texttt{BluetoothDevice} objects.
You can find your wireless device in that list and tell the Android OS to connect to that device. 
This device list is also used to fill a device choosing graphical dialog described before.

When devices get connected  it is able to receive a communication socket object and get a standard \texttt{java.io.InputStream} and \texttt{java.io.OutputStream} from that.
This is a final step of establishing the communication and it is able to transfer data over received stream objects.


The client RPC library is connected to these streams and it is able to communicate with a remote device over the wireless serial interface.
RPC calls may be started right after application gets connected to the Bluetooth input and output streams.


Now it is up to your imagination how to use and extend this remote embedded service.
You can implement any rich functionality and create whatever complex system you want or your hardware resources allow you.
This extensible service oriented embedded architecture allows you to do so.
