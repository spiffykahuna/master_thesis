%% LyX 2.0.3 created this file.  For more info, see http://www.lyx.org/.
%% Do not edit unless you really know what you are doing.
\documentclass[a4paper,10pt, times]{article}
\renewcommand{\rmdefault}{cmr}
\renewcommand{\sfdefault}{cmss}
\renewcommand{\ttdefault}{cmtt}
\renewcommand{\familydefault}{\sfdefault}
\usepackage[T1]{fontenc}
\usepackage[utf8]{inputenc}
\usepackage{geometry}
\geometry{verbose,tmargin=2cm,bmargin=2cm,lmargin=3.5cm,rmargin=2cm}

\makeatletter
%%%%%%%%%%%%%%%%%%%%%%%%%%%%%% User specified LaTeX commands.\usepackage[toc]{glossaries}
%\usepackage[estonian]{babel}
\usepackage[russian,estonian,english]{babel}
\usepackage{datetime}


\usepackage{textcomp} % Used for syntax highlighting. 
\usepackage{listings}
\usepackage{color}
\usepackage{palatino} 
\usepackage{setspace}
\usepackage{needspace} % line skipping

% images
\usepackage{graphicx}
\usepackage{wrapfig}


%\usepackage[printonlyused,withpage]{acronym}


\usepackage[bookmarks=true,
bookmarksnumbered=true,
bookmarksopen=false,
backref=page,
,colorlinks=true]{hyperref}
%\hypersetup{backref=true}
%\usepackage{backref}
%\usepackage{hyperref}
%\usepackage[colorlinks,bookmarks=false]{hyperref}
\usepackage{url}

% bibliography to table of contents
%\usepackage[nottoc,numbib]{tocbibind}


%\bibliography{references.bib}
% urls in bibliography
%\usepackage{natbib}
%\usepackage{babelbib}

% \usepackage[
%     backend=biber,
%     style=ieee,
%     sortlocale=de_DE,
%     natbib=true,
%     url=false, 
%     doi=true,
%     eprint=false,
%     backref=true
% ]{biblatex}
%\usepackage[style=ieee]{biblatex}
%\addbibresource{references.bib}



\usepackage{float}

% first paragraph indent
%\usepackage{indentfirst}
\usepackage{longtable}



%\input{macros/tables.tex}
%\usepackage{../resource/python}

\usepackage[acronym,footnote,toc]{glossaries}


\makeglossaries

%\newcommand{\estDate}[3]{\begin{otherlanguage}{estonian}\formatdate{#1}{#2}{#3} \end{otherlanguage}}

%\newcommand{\estToday}{\begin{otherlanguage}{estonian}\today \end{otherlanguage}}

%%
%% eestikeelsed kuupa"evad
%%
\def\@kuuaasta{\ifcase\the\month\or
  jaanuar\or veebruar\or märts\or aprill\or mai\or juuni\or
  juuli\or august\or september\or oktoober\or november\or detsember\fi
  \space \number\the\year}

\def\@aasta{\the\year}


\newcommand{\estDate}[3]{%
\begin{otherlanguage}{estonian}
#1.~
\ifcase #2 \relax
\or jaanuar
\or veebruar
\or märts
\or aprill
\or mai
\or juuni
\or juuli
\or august
\or september
\or oktoober
\or november
\or detsemberhttp://staff.ttu.ee/~alahe/ .
\fi
~ #3 .~a.
 \end{otherlanguage}}

\newcommand{\estToday}{\estDate{\the\day}{\the\month}{\the\year}}


%\input{macros/languages.tex}

%  some renamings
\AtBeginDocument{
% estonian language
%\renewcommand{\prefacename}{Sissejuhatus}
%\renewcommand{\abstractname}{Kokkuvõte}
%\renewcommand{\contentsname}{Sisukord}
%\renewcommand{\listfigurename}{Joonised}
%\renewcommand{\listtablename}{Tabelid}
%\renewcommand{\refname}{Viited}
%\renewcommand{\bibname}{Kirjandus}
%\renewcommand{\indexname}{Indeks}
%\renewcommand{\figurename}{Joonis}
%\renewcommand{\tablename}{Tabel}
%\renewcommand{\partname}{Osa}
%\renewcommand{\chaptername}{Peatükk}
%\renewcommand{\appendixname}{Lisa}
%\renewcommand{\enclname}{Lisa(d)}
%\renewcommand{\ccname}{Koopia(d)}
%\renewcommand{\headtoname}{Kellele}
%\renewcommand{\pagename}{Lk.}
%\renewcommand{\seename}{vt.}
%\renewcommand{\alsoname}{vt. ka}
%
% russian language
%
%\renewcommand{\prefacename}{Предисловие}
%\renewcommand{\abstractname}{Аннотация}
%\renewcommand{\contentsname}{Оглавление}
%\renewcommand{\listfigurename}{Список рисунков}
%\renewcommand{\listtablename}{Список таблиц}
%\renewcommand{\refname}{Литература}
%\renewcommand{\bibname}{Литература}
%\renewcommand{\indexname}{Предметный указатель}
%\renewcommand{\figurename}{Рис.}
%\renewcommand{\tablename}{Таблица}
%\renewcommand{\partname}{Часть}
%\renewcommand{\chaptername}{Глава}
%\renewcommand{\appendixname}{Приложение}
%\renewcommand{\enclname}{вкл.}
%\renewcommand{\ccname}{исх.}
%%\renewcommand{\headtoname}{вх.}
%\renewcommand{\pagename}{стр.}
%\renewcommand{\seename}{см.}
%\renewcommand{\alsoname}{см. также}
%
%
%\renewcommand{\contentsname}{Оглавление}
%
%\frenchspacing
%\setlength{\parskip}{20pt  plus 1pt minus 1pt}
\setlength{\parindent}{1cm}
}

\makeatother

\begin{document}

	%\newglossaryentry{H20}{name=water,description={The equation of wather}}
%\newglossaryentry{FPS}{name=fpsLabel,description={Frame per Second}}
%\newglossaryentry{A}{name=area,description={Area}}

\newacronym{FPS}{FPS}{Frame per Second}
\newacronym{H20}{H20}{The equation of wather}
\newacronym{FPGA}{FPGA}{Field-programmable gate array}
\newacronym{ASIC}{ASIC}{Application-specific integrated circuit}
\newacronym{SOA}{SOA}{Service-oriented architecture}
\newacronym{RPC}{RPC}{Remote procedure call}
\newacronym{JSON}{JSON}{JavaScript Object Notation}
\newacronym{XML}{XML}{Extensible Markup Language}
\newacronym{MCU}{MCU}{Microcontroller}



	
	\begin{titlepage}
	\begin{center}	
		\begin{minipage}[h]{0.01\linewidth}
			\begin{flushleft}	
				\includegraphics[scale=0.2]{../images/template/ttu_logo_2.jpg}
			\end{flushleft}		
		\end{minipage}
		\begin{minipage}[h]{0.98\linewidth}
			\begin{center}
				
				\textbf{TALLINN UNIVERSITY OF TECHNOLOGY}\\			
				Faculty of Information Technology \\			
				\textit{Department of Computer Engineering} \\			
		
			\end{center}
		\end{minipage}
	\end{center}
	\vspace{3cm}
	
	\begin{center}
		Denis Konstantinov \footnotesize \textsf{111615 IASM}
	\end{center}
	
	\vspace{5em}
	
	\begin{center} 
		\Large {Bitcoin mining on FPGA}\\
		\vspace{1em}
		\small {some description}\\	
	\end{center}
	
	\vspace{2em}
	
	\begin{center}		
		\textsf{Master thesis}
	\end{center}
	
	\vspace{2cm}
	
	% Author and supervisor	
	\begin{flushright}		
			%\large \emph{Esitatud:} & \quad \estDate{23}{10}{2008}  \\		
			%\large \emph{Submitted:} & \quad {\the\day}.{\the\month}.{\the\year}\\
			
			\emph{Supervisor:} \quad Thomas Hollstein \\
			\small{Professor at the Department of Computer Engineering /  Ph.D.}		
	\end{flushright}
	
	\vspace{\fill}
	\begin{center} 
		Tallinn \the\year
	\end{center}	
\end{titlepage}
		
	\clearpage\vspace*{\fill}

\section*{Author's Declaration} 


This work is composed by myself independently. All other authors' works, essential
states from literary sources and facts from other origins, which were used during the
composition of this work, are referenced.

\vspace{7em}
Signature of candidate: \hspace{10em}Date:

\vspace{\fill}
\clearpage



	
	\printglossaries
	
	\clearpage\vspace*{\fill}
\section*{Annotation}

Current work introduces conceptual approaches for implementing an extensible
service oriented client-server application on a small microcontroller.
This is a general-purpose transport and hardware independent embedded server
that uses remote procedure calls as primary communication protocol.
This server looks like remote service that could provide defined functions to
the client. \ldots

\vspace{\fill}
\clearpage
	\clearpage\vspace*{\fill}
\section*{Annotatsioon}
Käesolevas töös kirjeldatakse kuidas rakendada teenusorienteritud klient server arhitektuuri sardmikrokontrolleril.
See on üldotstarbeline transport ja riistvara sõltumatu rakendusserver,
mis kasutab kaugprotseduuri väljakutseid kommunikatsiooni pidamiseks.
Mikrokontrolleris jooksev programm näeb välja nagu serveri teenus,
mis pakkub kasutajale ettenähtud funktsionaalsust.

Antud töö eesmärk on teha uuring teenus orienteeritud arhitektuuride ja tehnoloogiate kohta 
ning analüsida nende kasutamist piiratud ressursidega sardsüsteemide realiseerimises.

All on toodud realiseeritud teenusorienteritud sardsüsteemi kirjeldus ja sellele vastav kliendi rakendus.

Serveri pool on tehtud STM32F1 pere ARM Cortex-M3 mikrokontrolleri baasil, kus jookseb FreeRTOS reaalaja operatsioonisüsteem.
Kliendi rakendus on mobiilsel platformil kasutatav programm, mis kasutab Javas kirjutatud teegi kaugprotseduuride väljakutsemiseks.

Klindi ja serveri vaheline kommunikatsioon toimub labi traadita Bluetooth kanali kasutades JSON-RPC protokolli.
Süsteemis on tehtud mõned funktsioonid, selleks et näidata antud klient-serveri arhitektuurilisi omadusi.


\vspace{\fill}
\clearpage
	
	\pdfbookmark{\contentsname}{contents}
	\tableofcontents
	
	
	\newpage
\section{Introduction}
\subsection{Outline}
\subsection{Contributions}



This is a test of acronyms  \gls{ASIC} \\

This is a test of code listings
\begin{listing}[H]
	\inputminted[linenos=true,
	fistline=32,
	firstnumber=32,
	lastline=60]{java}{../source/CoffeeMachineViaBlueTooth.java}
	\caption{Example of a listing.}
	\label{lst:example}
\end{listing}


At first here will be lots of words about this cruel world and how it was
changed in resent years.

Speech about reusable components.

Speech about small mobile devices that are everywhere.


Communication.

Interaction.



	
	\newpage
	\listoffigures
	\newpage
	\listoftables
	
	
	\bibliographystyle{ieeetr}
	\bibliography{../resource/references}
	

\end{document}