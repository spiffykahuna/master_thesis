\subsection{REST and RESTful services}

\subsubsection{What is the REST?}
Representational state transfer (REST) is a software design model for
distributed systems~\cite{Fielding2000}. This term was introduced in 2000 in the doctoral dissertation of
Roy Fielding, one of the principal authors of the Hypertext Transfer Protocol
(\gls{HTTP}) specification. REST uses a stateless, client-server,
cacheable communications protocol which is almost always the HTTP protocol. Its original
feature is to work by using simple HTTP to make calls between machines instead
of choosing more complex mechanisms such as CORBA, RPC or SOAP.

REST-style architectures conventionally consist of clients and servers.
Clients make requests to servers, servers process requests and return responses.
Requests and responses are built around the transfer of \textit{representations}
of \textit{resources}.
Author defines the resource as the key abstraction of information in
REST~\cite{Fielding2000}. It can be any information that can be named and
addressed: documents, images, non-virtual physical systems and services.
A representation of a resource is typically a document that captures the current or intended state of a resource.

Restful applications use HTTP requests to change a state of resource  \footnote{
all four CRUD(Create/Read/Update/Delete) operations)}: post data to create and/or update
resource, read data (e.g., make queries) to get current state of resource, and
delete data to delete existing resource.

REST does not offer security features, encryption, session management, QoS
guarantees, etc. But these can be added by building on top of HTTP, for example
username/password tokens are often used for encryption, REST can be used on
top of HTTPS (secure sockets)\cite{ws_techologies_state_of_the_art}.

\subsubsection{Key principles of REST}
REST is a set of principles that define how Web standards, such as HTTP and
\gls{URI}s, are supposed to be used.

The  five key principles of REST are\cite{rest_brief_intro}:
\begin{itemize}
  
 \item Give every “thing” an ID
 \item Link things together
 \item Use standard methods
 \item Resources with multiple representations
 \item Communicate statelessly
  
\end{itemize}

\paragraph{Give every “thing” an ID}  ~\\

Every resource need to be reachable and identifiable. You need to access it
somehow, therefore you need an identificator for the resource. World Wide Web
uses \gls{URI} identificators for that purpuse. Resource \gls{URI} could look like:

\begin{listing}[H]
\begin{minted}[frame=lines,
               framesep=2mm]{html}
http://example.com/customers/1234
http://example.com/orders/2007/10/776654
http://example.com/products/4554
http://example.com/processes/salary-increase-234 
http://example.com/orders/2007/11
http://example.com/products?color=green
\end{minted}
\caption{Resource identificator examples \cite{rest_brief_intro}}
\label{lst:uri_example}
\end{listing}

URIs identify resources in a global namespace. This means that this
identificator should be unique and there should not be another same URI.
This URI may reflect a defined customer, order or product and it might correspond
to database entry. \autoref{lst:uri_example} last two examples identify more
than one thing. They identify a collection of objects, which is the object
itself and require an identificator.

\paragraph{Link things together}  ~\\

Previous principle introduced an unigue global identificator for the resource.
Resource URI gives posibility to access the resource from different locations
and applications. Resources can be also linked to each other.
Listing \ref{lst:linked_uri_example} shows such scheme. Representation of an
order contains the information about this order and linked product and client resources.
This approach gives client an opportunity to change a state of client application by following linked
resources. After receiving order information client has two possibilities for
choice: to get product information or to fetch customer details. 

\begin{listing}
\begin{minted}[frame=lines,
               framesep=2mm]{xml}
<order self='http://example.com/orders/2007/10/776654' > 
   <amount>23</amount> 
   <product ref='http://example.com/products/4554' /> 
   <customer ref='http://example.com/customers/1234' /> 
</order> 
\end{minted}
\caption{Example of linked resources\cite{rest_brief_intro}}
\label{lst:linked_uri_example}
\end{listing}
 
The idea of links is a core principle of the Web~\footnote{The World Wide Web
(abbreviated as WWW or W3,[3] commonly known as the web), is a system of
interlinked hypertext documents accessed via the Internet.\cite{wikipedia:WWW}}

\paragraph{Use standard methods} ~\\

There should be a standard inteface for accessing the resource object.
REST relies on HTTP protocol, which has definitions of some standard request
methods:
GET, PUT, POST and DELETE.
\autoref{tbl:rest_http_api} describes standard actions on resource.

\begin{table}[h]
	\centering	
	\begin{tabularx}{\textwidth}{|X|X|X|X|X|}
		\hline
		\textbf{Resource} & 
		\textbf{GET}  	& 
		\textbf{PUT} 	&
		\textbf{POST} &
		\textbf{DELETE}
	    
	    \tabularnewline
		\hline
			\begin{sloppypar}
				\textbf{Collection URI, such as http://example.com/resources}
			\end{sloppypar} &
			\textbf{List} the URIs and perhaps other details of the collection's members.&
			\textbf{Replace} the entire collection with another collection.&
			\textbf{Create} a new entry in the collection. The new entry's URI is
			assigned automatically and is usually returned by the operation. &
			\textbf{Delete} the entire collection.
			
	    	\tabularnewline	    	
	    	\hline
	    	\begin{sloppypar}
				\textbf{Element URI, such as http://example.com/resources/item17} 
			\end{sloppypar} &
			\textbf{Retrieve} a representation of the addressed member of the
			collection, expressed in an appropriate Internet media type. &
			\textbf{Replace} the addressed member of the collection, or if it
			doesn't exist, \textbf{create} it. &			
			Not generally used. Treat the addressed member as a collection in its own
			right and \textbf{create} a new entry in it. &			
			\textbf{Delete} the addressed member of the collection.
	
	    \tabularnewline
		\hline	  
	\end{tabularx} 
	\caption{RESTful web API HTTP methods \cite{wikipedia:REST}}
	\label{tbl:rest_http_api}
\end{table}


\paragraph{Resources with multiple representations} ~\\
\paragraph{Communicate statelessly} ~\\


operations


CRUD table


