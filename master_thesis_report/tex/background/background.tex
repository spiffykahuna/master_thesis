\newpage
\section{Background}

Internet technology is the environment in which billions of people and trillions of
devices are interconnected in various ways.
As part of this evolution, Internet becomes the basic carrier
for interconnecting electronic devices – used in industrial
automation, automotive electronics, telecommunications
equipment, building controls, home automation, medical
instrumentation, etc. – mostly in the same way as the Internet came
to the desktops before. More and more devices getting connected to World Wide
Web. Variety of factors have influenced  this evolutions ~\cite{4221180}:
\begin{itemize}
\item The availability of affordable, high-performance, low-power
electronic components for the consumer devices. Improved technology can assist
building  advanced functionality into embedded devices and enabling new ways of
coupling between them.
\item Even low cost embedded devices have some wired or wireless interface to local area
networks of the Ethernet type. TCP/IP family protocols are becoming the standard
vehicle for exchanging information between networked devices.
\item The emergence of platform independent data interchange mechanisms based on
Extensible Markup Language (XML) data formatting gives lots of opportunities for developing high-level data interchange and
communication standards at the device level.
\item The paradigm of Web Services  helps to connect various
independent applications using lightweight communications. Clients that are
connected to the service and the service itself may be written using different
programming languages and be executed on different platforms.
\item Presence of Internet allows existing of small embedded controllers and
large production servers in the same network, with a possibility to change
information.
\end{itemize}


The integration of different classes of devices, which employ different
networking technologies, is still an open research area. One of the possible
solutions is the use of SOA software architecture design pattern.



\subsection{Service oriented architecture}
\begin{quote}
\textit{
Service-oriented computing is a computing paradigm that uses services as basic building blocks for
application development.}
~\cite {lws_milanovic.pdf}
\end{quote} 

The purpose of \gls{SOA} is to allow easy cooperation of a large number of
computers that are connected over a network.
Every computer can run one or more services, each of them implements one
separate action. This may be a simple business task. Clients can make calls
and receive required data or post some event messages.

Services are self-describing and open components. There is a service interface,
that is based on the exchange of messages and documents using standard formats.
Interface internals ( operating system, hardware platform, programming language)
are hidden from client. Client uses only a service specification scheme, also
called contract. Consumers can get required piece of functionality by mapping problem
solution steps to a service calls. This scheme provides quick access and easy integration of software components.


Service architecture have been successfully adopted in
business environments. Different information systems, that were created inside
companies for automation of business processes,  are now turned into services
which may easily interact with each other. For example, Estonian government uses
services to transmit data between information systems of different departments.
There are also some free services available. Some Internet search companies like
Google, Bing, Yandex provide lots of alternatives how to retrieve data without
using regular browser( search , geolocation and maps,  spell check \gls{API}s )

There are available many technologies which can be used to implement
\gls{SOA}~\cite{wikipedia:SOA}:

\begin{itemize}
  \item \label{itm:soa_technologies} \textbf{Web Services}
  \item{\textbf{SOAP}} Simple Object Access Protocol, is a protocol specification for exchanging structured information in the implementation of Web Services in computer networks.
  \item{\textbf{RPC}} Remote procedure call is an inter-process communication that allows
  a computer program to cause a subroutine or procedure to execute in another address space (commonly on another computer on a shared network) without the programmer explicitly coding the details for this remote interaction. 
  \item{\textbf{REST}} Representational state transfer is a style of software architecture for distributed systems such as the World Wide Web. REST has emerged as a predominant web API design model.
  \item{\textbf{DCOM}} Distributed Component Object Model is a proprietary Microsoft technology for communication among software components distributed across networked computers. 
  \item{\textbf{CORBA}}  Common Object Request Broker Architecture enables separate pieces of software written in different
    languages and running on different computers to work with each other like a single application or set of services. Web services
  \item{\textbf{DDS}} Data Distribution Service for Real-Time Systems (DDS) is an Object
  Management Group (OMG) machine-to-machine middleware standard that aims to
  enable scalable, real-time, dependable, high performance and interoperable data exchanges between publishers and subscribers. 
  \item{\textbf{Java RMI}} Java Remote Method Invocation is a Java API that performs the
  object-oriented equivalent of remote procedure calls (RPC), with support for direct transfer of serialized Java objects and distributed garbage collection.
  \item{\textbf{Jini}}  also called Apache River, is a network architecture for the construction of distributed systems in the form of modular co-operating services.
  \item{\textbf{WCF}} The Windows Communication Foundation (or WCF), previously known as
  "Indigo", is a runtime and a set of APIs (application programming interface)
  in the .NET Framework for building connected, service-oriented applications.
  \item{\textbf{Apache Thrift}} is used as a remote procedure call (RPC) framework and
  was developed at Facebook for "scalable cross-language services development".
  \item \ldots
\end{itemize}

This list can be continued. Most of these technologies are inspired by idea
of \gls{RPC}. An \gls{RPC} is initiated by the client, which sends a request
message to a known remote server to execute a specified procedure with specified
parameters. The remote server sends a response to the client, and the application continues its process.
This idea is described in details in \autoref{sec:rpc}.



Web Services are the most popular technology for implementing
service-oriented software nowadays. Next section will focus on this framework
and on the main features that any \gls{SOA} implementation should have.

\subsection{Web Services architecture}
\subsubsection{Web Services Model}
The Web Services architecture is based on the interactions between three
roles \cite{Kreger2001-WSC}:
service provider, service registry and service requestor. 
This integration has of three operations: publish, find and
bind. The service provider has an implementation of service. Provider defines a
service description and publishes it to a service requestor or service registry.
The service requestor uses a find operation to retrieve the service
description locally or from the service registry and uses the service description to bind with the
service provider and invoke or interact with the Web service implementation.
\autoref{fig:ws_model} illustrates these service roles and their operations.


\begin{center}
 \begin{figure}[H]
	\includegraphics[width=\textwidth]{../images/background/ws_model.png}
	\caption{Web Services roles, operations and artifacts \cite{Kreger2001-WSC}}
	\label{fig:ws_model}
 \end{figure}
\end{center}

Service registry is a place where service providers can publish
descriptions of their services. Service requestors can find service descriptions
and get binding information from them. Binding can be static and dynamic.
Registry is needed more for dynamic binding where client can get service info at
the runtime, extract necessary functional methods and execute them on the
server. During static binding service description may be directly delivered to
the client at the development phase, for example using usual file, \gls{FTP}
server, Web site, email or any other file transfer protocol.
There are also available special protocols, named Service discovery protocols
(SDP), that allow automatic detection of devices and services on a network. One
of them is the \gls{UDDI} protocol, which is also was mentioned on
\autoref{fig:ws_model}. \gls{UDDI} is shortly described in
\nameref{sec:ws_protocol_stack} section.

\paragraph{\textbf{Artifacts of a Web Service}}
\newline
Web service consists of two parts~\cite{Kreger2001-WSC}:
\begin{itemize} 
\item \label{itm:service_description_artifact} 
\textbf{Service Description}~~~The
service description contains the details of the interface and implementation of the service. This includes its data types, operations, binding
information and network location. There could also be a categorization and
other metadata about service discovery and utilization. It may contain some
Quality of service (QoS) requirements. 

 \item \textbf{Service}
 ~~~This is the implementation of a service - a software module deployed on network accessible platforms provided by the service provider.
 Service may also be a client of other services. Implementation details
 are encapsulated inside a service, and client does not know the details how
 server processes his request.
\end{itemize}



\subsubsection{Web Services Protocol Stack}
\label{sec:ws_protocol_stack}

WS architecture uses many layered and interrelated technologies.
\autoref{fig:ws_protocol_stack} provides one illustration of some of these technology families.

\begin{center}
 \begin{figure}[h]
	\includegraphics[width=\textwidth]{../images/background/ws_protocol_stack.png}
	\caption{Web Services Architecture Stack \cite{ws_arch} }
	\label{fig:ws_protocol_stack}
 \end{figure}
\end{center}

We can describe these different layers as follows:
\begin{itemize}
  \item \textbf{Communications} - This layer represents a transport between
  communication parties( service provider, client, service registry). This layer
  can be any network protocol like: \gls{HTTP}, \gls{FTP}, \gls{SMTP} or any
  other suitable transport protocol. If Web service is used in the Internet, the
  transport protocol in most cases will be \gls{HTTP}. In internal networks there is the opportunity to agree upon the use of alternative
network technologies.

\item \textbf{Messages} - In order to communicate with a service, client should
send a message. Messages are \gls{XML} documents with different structure.
\gls{SOAP} protocol defines how these messages should be structured.
\gls{SOAP} is implementation independent and may be composed using any
programming language. Protocol specification and message descriptions can be
found in document SOAP Version 1.2 Part 1: Messaging Framework (Second
Edition)\cite{soap_protocol_spec}.

\item \textbf{Descriptions} - This layer contains the definition of service
interface (see also \autoref{itm:service_description_artifact}). Web Services
use \gls{WSDL} language for describing the functionality offered by a service.
\gls{WSDL} file is the contract of service, which contains information about how
the service can be called, what parameters it expects, and what data structures it returns.
It is similar to method signatures in different programming languages.

\item \textbf{Processes} - This part contains specifications and protocols
about how service could be published and discovered. Web services are meaningful
only if potential users may find information sufficient to permit their execution.
Service as a software module has its own lifecycle, it needs to be deployed and deleted somehow.
Traditional Web Services use \gls{UDDI}  mechanism to register and locate web
service applications. \gls{UDDI} was originally proposed as a core Web service
standard.

\item \textbf{Security} - Threats to Web services include threats to the host
system, the application and the entire network infrastructure. To solve problems
like authentication, role-based access control, message level security there is
need for a range of XML-based security mechanisms.

Web services architecture uses WS-Security\footnote{WS-Security (Web Services
Security, short WSS) is an extension to SOAP to apply security to web services.
It is a member of the WS-* family of web service specifications and was
published by OASIS(Organization for the Advancement of Structured Information
Standards, https://www.oasis-open.org/). \cite{wikipedia:WS-Security}}
~protocol to solve security problems.
This protocol specifies how \gls{SOAP} messages may be secured.
 


\item \textbf{Management} -
Web service management tasks are~\cite{ws_arch}: monitoring, controlling,
reporting of service state information. 

Web services architecture uses WS-Management~\cite{ws_security_promo} protocol
for management of entities such as PCs, servers, devices, Web services and other
applications. WS-Management has ability to discover the presence of management resources,
control of individual management resources, subsrcibe to events on resources,
execute specific management methods.

WS-Management was created by DMTF( Distributed Management Task Forse, Inc.,
http://www.dmtf.org/) organization, which is creating stantards for managing
the enterprise level systems. Organisation members are largest hardware and
sortware corporations like Broadcom, Cisco, Fujitsu, Hewlett-Packard, IBM, Intel,
Microsoft, Oracle. DTMF standards promote multi-vendor interoperability, which
is great for the integration between different IT systems. 

\end{itemize}

Most of mentioned protocols are recommended by W3C Consortium and are production
standards. Lots of \gls{SOA} information systems use WS-* protocols for
enterprise level services. Mostly these protocols are based on \gls{XML} and
\gls{SOAP}. One example of such protocol is the \gls{WSDL} service description.

\subsubsection{Service Description and Service Contract}
\label{sec:ws_service_contract}
\gls{WSDL} file is an \gls{XML} document which has specification of  service contract.
As it was mentioned earlier(\autoref{itm:service_description_artifact}) contract should be shipped with a component and should  tell the
client what input does service expect, and what output it will produce if specified input conditions are met. 
Contract may be a primary specification and it should be enough for a client to start using a service.
This is similar to library header file in C language. 
You have a ready and compiled library shipped with a header file, where are all
method declarations and definitions of data structures.
If header file is verbose enough, there is no need to use the documentation. You can place this component into your system very easily.



Regular \gls{WSDL} document contains some necessary elements
~\cite{wsdl_language_spec, wikipedia:WSDL}: Service, Endpoint, Binding,
Interface and Message Types. \autoref{tbl:wdsl_members} describes them in
details.


\begin{table}[h]
	\centering	
	\begin{tabular}[h]{|l|p{10cm}|}
		\hline
		\textbf{WSDL 2.0 Term} & 
		\textbf{Description} 	
	    	\tabularnewline
		\hline
			Service &
			The service element describes \textit{where} to access the service.\newline A
			WSDL 2.0 service specifies a single interface that the service will support,
			and a list of endpoint locations where that service can be accessed.
	    	\tabularnewline	    	
	    	\hline
			Endpoint & 
			Defines the address or connection point to a Web service.
			It is typically represented by a simple HTTP URL string.			
			Each endpoint must also reference a previously defined binding to indicate			
			what protocols and transmission formats are to be used at that endpoint.				
	    	\tabularnewline

	    	\hline
			Binding &		
			Specifies concrete message format and transmission protocol details			 
			for an interface, and must supply such details for every operation			 
			and fault in the interface.
	    	\tabularnewline

	    	\hline
			Interface &
			Defines a Web service, the operations that can be performed,			
			and the messages that are used to perform the operation.			
			Defines the abstract interface of a Web service as a set of abstract \textit{operations},			
			each operation representing a simple interaction between the client and the service.			
			Each operation specifies the types of messages that the service can send or receive as part of that operation.			
			Each operation also specifies a message exchange \textit{pattern} that indicates the sequence in which the associated messages are to be transmitted between the parties. 
	    	\tabularnewline

		\hline
			Message Types &
			The types element describes the kinds of messages that the service will send and receive.			
			The XML Schema\footnotemark ~language (also known as XSD) is used (inline or
			referenced) for this purpose.
	    	\tabularnewline
		\hline	  
	\end{tabular} 
	\caption{Objects in WSDL 2.0~\cite{wsdl_language_spec, wikipedia:WSDL} }
	\label{tbl:wdsl_members}
\end{table}
 
\footnotetext{XML schema is the XML document, that
			specifies structure of other XML document and describes data types and constraints, that other document might have. 
			You can create a schema for necessary XML data structure and verify if processed message corresponds to schema you have already defined}


Document may also contain optional element named \textit{documentation}.
There may be human readable service documentation, with   purpose and use of the service, the meanings of all messages, constraints their use, and the sequence in which operations should be invoked.
To be documentation more complete, you may specify an external link to any additional documentation.


You can see the example of WSDL service contract in the appendix 
\autoref{sec:appendix_service_contracts}. This example is from the official
\gls{WSDL} standard ~\cite{wsdl_language_spec}. It describes a hotel reservation
service, where you can book you a room in a fictional hotel named GreatH. For simplicity it describes
only one method - the \textit{opCheckAvailability} operation. This description
is quite verbose to understand what it is about. There is input and output
object type declaration. It also has an output error response declaration and
if some error occures during client request processing, server should send a
message of specified kind.
These \gls{XML} object types are declared in different \textit{namespaces} ( see
xmlns:*	declarations at the start of \gls{WSDL} document). This gives an ability
to group domain types into one separate file and your main description
file would not be overcrowded.


\gls{WSDL} definition of the service does not contain any additional
information about service hosting company and its products. At the moment when
you get a service contract, you already know what company provides this service and what
for this service was made. There is assumption that service provider somehow
gave you this service contract. Another possibility to get the service
description is to use a special \textit{directory} or catalog, where you can
find all information about company you are dealing with. Web Services
architecture include \gls{UDDI} mechanism for that particular purpose.

Official \gls{UDDI} Version 3.0.2 Technical Specification draft~\cite{uddi_spec}
defines \gls{UDDI} as follows:
\begin{quotation}
\textit{
 The focus of Universal Description Discovery & Integration (UDDI) is the
definition of a set of services supporting the description and discovery of (1) businesses,
organizations, and other Web services providers, 
(2) the Web services they make available,
 and (3) the technical interfaces which may be used to access those services.
 Based on a common set of industry standards, including HTTP, XML, XML Schema, and SOAP,
 UDDI provides an interoperable, foundational infrastructure for a Web services-based software environment
 for both publicly available services and services only exposed internally within an organization.
}
\end{quotation}

\gls{UDDI} mechanism uses \gls{SOAP} messages for client-server communication. 
Service provider publishes the \gls{WSDL} to \gls{UDDI} registry and client can
find this service by sending messages to the registry ( see also 
\autoref{fig:ws_model}). \gls{UDDI} specification defines the communication
protocol between \gls{UDDI} registry and other parties.


\gls{UDDI} registry is a storage directory for various service contracts, where
lots of companies hold their service descriptions. \gls{WSDL} contracts may 
also be published on a company website using direct link to the \gls{WSDL} file,
but \gls{UDDI} contains them all in one place with the ability to search and
filter.
You have a choice and there is a possibility to find most suitable service
from all provided companies and services.



\subsubsection{Advantages and disadvantages of WS-* standards}
\label{sec:adv_and_disadv_of_ws_standards}
Usage of WS-* technologies gives some
benefits~\cite{ws_techologies_state_of_the_art}:

\begin{itemize}
  \item Reusability
  \item Interoperability and Portability
  \item Standardized Protocols
  \item Automatic Discovery
  \item Security
\end{itemize}

WS-* uses the \gls{HTTP} protocol as transport medium for exchanging messages
between web services. \gls{SOAP} messages can be transfered using another
protocol (\gls{SMTP}, \gls{TCP}, or \gls{JMS}), which can be more suitable to
your system environment than \gls{HTTP}.

One another fundamental characteristic of web services is the service
description and contract design. Contract specification gives you the ability to
reuse service functionality in many different and separate applications.
Contract in Web Services is general standardized description of a service in
universal data format (\gls{XML} and \gls{WSDL}), that is platform and programming language
independent. Service description is only the interface and the implementation
of that interface may be unknown by the service client. There is possibility to
transparently change (totally or partially) implementation details of a service.


The main reason why Web Services standards are bad in context of embedded
systems is the performance.
Web services impose additional overhead on the server since they require the
server to parse the \gls{XML} data in the request. Web Services use \gls{SOAP}
messages, which are structured \gls{XML} data, for client-server communication.
Some experiments~\cite{1182978, 5470528} show that performance of SOAP transfer
is more than 5 times slower compared to others \gls{SOA} implementations, like
CORBA~\footnote{The Common Object Request Broker Architecture (CORBA) is a
standard defined by the Object Management Group (OMG) that enables software
components written in multiple computer languages and running on multiple
computers to work together (i.e., it supports multiple
platforms).\cite{wikipedia:CORBA}} or custom made protocol messages. If you
start reading WS-* standarts one by one, you will ensure that all they are
interconnected using idea of \gls{XML}, \gls{SOAP} and \gls{HTTP}. 


Another statement against Web Services is that these standards are too
complicated and their documentation is hard to
understand. Document ~\cite{dguinard-rest-vs-ws} contains a use case survey
about two most common implementations of SOA: Web Services and \gls{REST}. Most of people
in the survey agreed that WS-* standards is not easy to learn and adapt. WS-*
suits better for highly integrated business solutions, but not for simple
applications, with atomic functionality.

WS-* standarts rely on each other and to implement a small web service with
few features you will need to dive into all WS standards. This is at least 783
pages of not just text, but a techical specifications~\cite{ws_pagecount}.
Surely, Web services have good ideas, but this technology is promoted and
developed by large corporations like Microsoft, IBM, Oracle, who are not
intrested in simple and lightweight solutions, because they need to utilize
their thousands of developers and earn money( document ~\cite{ws_pagecount} says
that most of WS-* specifications are hosted by Microsoft and  OASIS
organisation, which foundational sponsors are IBM and Microsoft).
It seems that Web services are trying to solve every business problem and there
is a \textit{WS-problem} standart for it.


This topic described Web Service architecture features. There are lots of
useful principles like portability, interface description and message exchange
patterns, but the WS-* implementation is not suitable for resource-constrained
hardware.


The \gls{REST} has some advantages over WS-*. Next section will shortly
describe the main principles of \gls{REST} approach. 






\subsection{REST and RESTful services}

\subsubsection{What is the REST?}
Representational state transfer (REST) is a software design model for
distributed systems~\cite{Fielding2000}. This term was introduced in 2000 in the doctoral dissertation of
Roy Fielding, one of the principal authors of the Hypertext Transfer Protocol
(\gls{HTTP}) specification. REST uses a stateless, client-server,
cacheable communications protocol which is almost always the HTTP protocol. Its original
feature is to work by using simple HTTP to make calls between machines instead
of choosing more complex mechanisms such as CORBA, RPC or SOAP.

REST-style architectures conventionally consist of clients and servers.
Clients make requests to servers, servers process requests and return responses.
Requests and responses are built around the transfer of \textit{representations}
of \textit{resources}.
Author defines the resource as the key abstraction of information in
REST~\cite{Fielding2000}. It can be any information that can be named and
addressed: documents, images, non-virtual physical systems and services.
A representation of a resource is typically a document that captures the current or intended state of a resource.

Restful applications use HTTP requests to change a state of resource  \footnote{
all four CRUD(Create/Read/Update/Delete) operations)}: post data to create and/or update
resource, read data (e.g., make queries) to get current state of resource, and
delete data to delete existing resource.

REST does not offer security features, encryption, session management, QoS
guarantees, etc. But these can be added by building on top of HTTP, for example
username/password tokens are often used for encryption, REST can be used on
top of HTTPS (secure sockets)\cite{ws_techologies_state_of_the_art}.

\subsubsection{Key principles of REST}
REST is a set of principles that define how Web standards, such as HTTP and
\gls{URI}s, are supposed to be used.

The  five key principles of REST are\cite{rest_brief_intro}:
\begin{itemize}
  
 \item Give every “thing” an ID
 \item Link things together
 \item Use standard methods
 \item Resources with multiple representations
 \item Communicate statelessly
  
\end{itemize}

\paragraph{Give every “thing” an ID}  ~\\

Every resource need to be reachable and identifiable. You need to access it
somehow, therefore you need an identificator for the resource. World Wide Web
uses \gls{URI} identificators for that purpuse. Resource \gls{URI} could look like:

\begin{listing}[H]
\begin{minted}[frame=lines,
               framesep=2mm]{html}
http://example.com/customers/1234
http://example.com/orders/2007/10/776654
http://example.com/products/4554
http://example.com/processes/salary-increase-234 
http://example.com/orders/2007/11
http://example.com/products?color=green
\end{minted}
\caption{Resource identificator examples \cite{rest_brief_intro}}
\label{lst:uri_example}
\end{listing}

URIs identify resources in a global namespace. This means that this
identificator should be unique and there should not be another same URI.
This URI may reflect a defined customer, order or product and it might correspond
to database entry. \autoref{lst:uri_example} last two examples identify more
than one thing. They identify a collection of objects, which is the object
itself and require an identificator.

\paragraph{Link things together}  ~\\

Previous principle introduced an unigue global identificator for the resource.
Resource URI gives posibility to access the resource from different locations
and applications. Resources can be also linked to each other.
Listing \ref{lst:linked_uri_example} shows such scheme. Representation of an
order contains the information about this order and linked product and client resources.
This approach gives client an opportunity to change a state of client application by following linked
resources. After receiving order information client has two possibilities for
choice: to get product information or to fetch customer details. 

\begin{listing}[H]
\begin{minted}[frame=lines,
               framesep=2mm]{xml}
<order self='http://example.com/orders/2007/10/776654' > 
   <amount>23</amount> 
   <product ref='http://example.com/products/4554' /> 
   <customer ref='http://example.com/customers/1234' /> 
</order> 
\end{minted}
\caption{Example of linked resources\cite{rest_brief_intro}}
\label{lst:linked_uri_example}
\end{listing}
 
The idea of links is a core principle of the Web~\footnote{The World Wide Web
(abbreviated as WWW or W3,[3] commonly known as the web), is a system of
interlinked hypertext documents accessed via the Internet.\cite{wikipedia:WWW}}

\paragraph{Use standard methods} ~\\

There should be a standard inteface for accessing the resource object.
REST relies on HTTP protocol, which has definitions of some standard request
methods:
GET, PUT, POST and DELETE.
\autoref{tbl:rest_http_api} describes standard actions on resource.

\begin{table}[h]
	\centering	
	\begin{tabularx}{\textwidth}{|X|X|X|X|X|}
		\hline
		\textbf{Resource} & 
		\textbf{GET}  	& 
		\textbf{PUT} 	&
		\textbf{POST} &
		\textbf{DELETE}
	    
	    \tabularnewline
		\hline
			\begin{sloppypar}
				\textbf{Collection URI, such as \url{http://example.com/resources}}
			\end{sloppypar} &
			\textbf{List} the URIs and perhaps other details of the collection's members.&
			\textbf{Replace} the entire collection with another collection.&
			\textbf{Create} a new entry in the collection. The new entry's URI is
			assigned automatically and is usually returned by the operation. &
			\textbf{Delete} the entire collection.
			
	    	\tabularnewline	    	
	    	\hline
	    	\begin{sloppypar}
				\textbf{Element URI, such as \url{http://example.com/resources/item17}} 
			\end{sloppypar} &
			\textbf{Retrieve} a representation of the addressed member of the
			collection, expressed in an appropriate Internet media type. &
			\textbf{Replace} the addressed member of the collection, or if it
			doesn't exist, \textbf{create} it. &			
			Not generally used. Treat the addressed member as a collection in its own
			right and \textbf{create} a new entry in it. &			
			\textbf{Delete} the addressed member of the collection.
	
	    \tabularnewline
		\hline	  
	\end{tabularx} 
	\caption{RESTful web API HTTP methods \cite{wikipedia:REST}}
	\label{tbl:rest_http_api}
\end{table}

All four \gls{CRUD} operations may be done with these HTTP methods. It is
possible to manage a whole lifecycle using only these methods. Client of the
service should only implement   the default application protocol (HTTP)
in correct way.

\paragraph{Resources with multiple representations} ~\\

When client gets representation of a resource using GET method he does not know
which data format will server return. REST as architectural method does not
provide any special standard for resource representation. How can client and
server could make an agreement about data format they will use?


HTTP protocol help to solve these problems again. 
It has a special field that defines the operating parameters of an HTTP
transaction. Field \textif{Accept} in the header of HTTP request message
specifies content type that is acceptable for the response~\cite{http-rfc}.

Such request header could look like this:

\begin{listing}[H]
\begin{minted}[frame=lines,
               framesep=2mm]{html}
GET /images/567 HTTP/1.1
Host: example.com
Accept: image/jpeg
\end{minted}
\caption{Request for a representation of resource in a particular format}
\label{lst:http_accept_example}
\end{listing}

This means that client expects that representation of a resource having
identificator \url{http://www.example.com/images/567} should be in
\textit{image/jpeg} format. Both client and server should be aware of such
format, whole system may be designed around any special format. 
There could be also an another representation of same resouce( the same image),
for example \textit{image/png} or \textit{image/bmp} formats, that server could
send according to received request. 

Not only outgoing data format could be specified. Server can also consume data
in specific formats (there are different specific header fields for this, for
example \textit{Content-Type}).

Using multiple representations of resources helps to connect more possible
clients to the system.

\paragraph{Communicate statelessly} ~\\

Stateless communication helps to design more scalable systems.
RESTful server does not  keep any communication context. Each incoming  request
is new for the server and there should be enough information to necessary to
understand the request\cite{Fielding2000}.

Server could contain all the information about each connected client, but it
requires a meaningful amount of resources. Server need to keep and control the
current state of application for every client, which locks the server resources
while client is not active (opened connections, memory, data integrity locks).

There is no need for keeping using all these resources if the application state
is on the client side. Client controls the flow of the application and
changes its state by making requests to server. Server does not make any work
while clients are not sending requests, it starts to work only on demand.

You can easilly switch between different servers if there is no any client
context on the server side. Imagine a system with some amount of application
servers and load balancer server in it. All application servers run the same
application. While some servers are making hard work, another servers may be
idle.
Load balancer have a posibility to route new incoming requests to a server,
which has a smallest workload, because of the absence of application context
between client and server. Even parallel request of the same client could be
handled by different servers. Such system become more scalable and new server
nodes can be easilly added or removed.

The main disadvantage of such approach is the decrease of network performance.
Client needs to repeat sending the same session data on every new request,
because that context data cannot be stored on the server. 


\subsubsection{Implementation constraints}

The REST architectural style can be described by the following six constraints
applied to the elements in this architecture\cite{Fielding2000}:
\begin{enumerate}
  \item \textbf{Client-Server}
  Separation of concerns is the principle behind the client-server constraints.
  User interface is separated from data storage and has improved portability.
  Server side does not aware of client application logic and server tasks may be
  more optimized and independent.
  \item \textbf{Stateless}
  This constraint reflects a design trade-off of keeping session
  information about each client on the server. Stateless method does not keep
  any application context on the server and allows to build more scalable server
  components. Each new request contains enough information to process it, but
  such technique increases network traffic by sending repeating session
  invormation over the network again and again.
  \item \textbf{Cache}
  This constraint improves network efficiency. The data in the server response
  may be marked as cacheable or non-cacheable. Client is able to store cachable
  data on its side and reuse it, if it needs to send the same request again.
  Frequently changed data should not be marked as cacheable in order to provide
  data integrity.
  \item \textbf{Uniform Interface}
  System becomes more universal if the interface of all components is the same.
  You can add new and replace existing componens more easilly. Component
  implementations are decoupled from the services they provide, system
  components are more independent. 
  \item \textbf{Layered System} style allows an architecture to be composed of hierarchical
  layers, that separate knowledge between components from different layers.
  Components in each layer do not know the structure of a whole system, but can
  only communicate to each other and with neighbour layers through a specified
  interface. Layers can be used to encapsulate legacy services and to protect
  new services from legacy clients. Structures inside a layer may be
  transparently changed.
  \autoref{fig:Uniform-Layered-Client-Cache-Stateless-Server} shows such a
  complex layered system.
  
  The primary drawback of layered systems is that they add overhead and latency
  to the processing of data. Every additional layer requires new amount of
  resources (which could be shared using common access, but layers are
  separated and they don't) and reduce communication performance by introducing
  new bottleneck at the layer boundaries.
  \item \textbf{Code-On-Demand}
  This is an optiona constraint. REST allows client functionality to be extended
  by downloading and executing code in the form of applets or scripts. Not all
  features on a client side may be implemented. Your system could download
  execution instructions from the server. This is how JavaScript and Java
  applets work in the Web browser.
 
\end{enumerate}




\begin{center}
 \begin{figure}[h]
	\includegraphics[width=\textwidth]{../images/preliminaries/Uniform-Layered-Client-Cache-Stateless-Server.png}
	\caption{Uniform-Layered-Client-Cache-Stateless-Server \cite{Fielding2000} }
	\label{fig:Uniform-Layered-Client-Cache-Stateless-Server}
 \end{figure}
\end{center}

\subsubsection{Summary}
Notion of independence was mentioned above in this section quite many times.
This is because REST  in itself is a high-level style that could be implemented
using many different technologies. It was initially described in the context
of HTTP, but it is not limited to that protocol.
 RESTful applications maximize the use of the existing, well-defined interface
and other built-in capabilities provided by the chosen network protocol,
and minimize the addition of new application-specific features on top of
it\cite{rest_brief_intro}.




\subsection{Remote procedure calls and *-RPC}
\label{sec:rpc}
Many distributed systems are based on explicit message exchange
between processes. If you see a list of SOA techologies ( provided above, see
\autoref{itm:soa_technologies}) you can find that many of these techologies use
RPC within them. Some of them don't, for example REST described in  previous
section, it uses different resource oriented approach(resources which
the client can consume). Other majority of techologies are message oriented and
use messages for \gls{IPC}. In RPC messages are sent between client and server
to call mathods and receive results. 

\subsubsection{RPC in details}
Remote procedure calls have become a de facto standard for communication
in distributed systems\cite{tanenbaum07}. The popularity of the model is due to
its apparent simplicity.
This section gives a brief introduction to RPC and the problems in there.

Only one figure is enough to describe RPC(see \autoref{fig:rpc_call}).

% \begin{center}
%  \begin{figure}[h!]
% 	\includegraphics[width=\textwidth]{../images/preliminaries/rpc_diagram.png}
% 	\caption{Principle of RPC between a client and server program }
% 	\label{fig:rpc_call}
%  \end{figure}
% \end{center}
% \includepdf[ angle=90, addtolist={ 1 , figure , {Principle of RPC between a client and server program } , {fig:rpc_call} }
% ]{../images/preliminaries/rpc_diagram.pdf}

% \begin{sidewaysfigure}[h]
%     \includegraphics{../images/preliminaries/rpc_diagram.eps}
%     \caption{Property profile of the diverse library compared to the compound pool.}
%     \label{fig:PropProf}
% \end{sidewaysfigure}


\begin{sidewaysfigure}
\centering
\scalebox{0.4}
{\includegraphics{../images/preliminaries/rpc_diagram.png}}
\caption{Principle of RPC between a client and server program}
\label{fig:rpc_call}
\end{sidewaysfigure}


When a process on client machine calls a
procedure on server machine, the calling process on client is suspended, and
execution of the called procedure takes place on server. Information can be
transported from the caller to the callee in the parameters and can come back in the procedure result.
No message passing at all is visible to the programmer. Programmer operates only
with method calls.

The idea behind RPC is to make a remote procedure call look as much as possible
like a local one. In other words, we want RPC to be transparent—the calling
procedure should not be aware that the called procedure is executing on a different
machine or vice versa\cite{tanenbaum07}. To achieve this transparency special
software modules are used. They are called \textbf{stubs}. The main purpose
of a stub is to handle network messages between client and server. 

Whole RPC call process could be described using words like this: 
\begin{enumerate}
\item The client procedure calls the client stub on the client machine.
\item The client stub builds a message and sends it over the network to remote
machine using local operating system(OS).
\item The remote  OS  receives the message from the network and gives
this message to the server stub. Server stub unpacks the parameters and calls
the server.
\item The server does the work and returns the result to the server stub.
\item The server stub packs result in a message and sends it to client using
network and underlying OS.
\item The client’s OS gives the message to the client stub.The stub unpacks the result and returns to the client.
\item Client receives procedure result and continues his processing.
\end{enumerate}

Modern software tools help to make this process very easy. Here
is the real world example of the small RPC system(written using Python
programming language) that proves this\cite{xmlrpclib_python_example}:

\begin{listing}[H]
\begin{minted}[frame=lines,
               framesep=2mm]{python}
import xmlrpclib
from SimpleXMLRPCServer import SimpleXMLRPCServer

def is_even(n):
    return n%2 == 0

server = SimpleXMLRPCServer(("example.com", 8000))
print "Listening on port 8000..."
server.register_function(is_even, "is_even")
server.serve_forever()
\end{minted}
\caption{RPC server example (Python and xmlrpclib)}
\label{lst:rpc_server_python_example}
\end{listing}

Code is quite verbose\footnote{Python programming language has its own
philosophy, called The Zen of Python. It has some statements how software
should be designed. Two statements in the Zen of Python that are related to
this example are: \begin{quote}\textit{Simple is better than
complex.}\end{quote}  and \begin{quote}\textit{Readability counts.}\end{quote}}
and people, who are not familiar with Python, could understand it.
You create a server using the special RPC server implementation from
\textit{xmlrpclib} library. You specify a remote host and a port number in a
object constructor. When server object is created you need to specify remote
methods, which may be executed. Example above uses small even check method.
Server registers the methods and starts to wait for incoming calls.

Client implementation for the corresponding server looks like:
\begin{listing}[H]
\begin{minted}[frame=lines,
               framesep=2mm]{python}
import xmlrpclib

proxy = xmlrpclib.ServerProxy("http://example.com:8000/")
print "3 is even: %s" % str(proxy.is_even(3))
print "100 is even: %s" % str(proxy.is_even(100))
\end{minted}
\caption{RPC client example (Python and xmlrpclib)}
\label{lst:rpc_client_python_example}
\end{listing}

Actually, it is more shorter and simpler than server code. You just specify a
remote server object and receive a proxy object. Then you can use this proxy like a usual
local object. This creates an illusion that you are not going anywhere for the
result and working with a usual objects in the code. Proxy object may be passed
as a parameter to a function and be used in that function without knowledge
where it was came from.


This idea is simple and elegant, but there are exist some problems. First of
all, calling and called procedures run on different machines and are executed in
different address spaces, which introduce additional complexity in passing
parameters and results between client and server.
Finally, both machines can independently crash, therefore special error
handling mechanism is required. System developers must deal with such failures
without knowing about remote procedure was actually invoked or not.

As long as the client and server machines are identical and all the parameters
and results are scalar types, such as integers, characters, and Booleans, this model
works fine. 
However, in a large distributed system, it is common that multiple
machine types are present.
Each machine often has its own representation for
numbers, characters, and other data items\cite{tanenbaum07}.

There are several representations of character data used in computer systems:
one byte characters(ASCII\footnote{ American Standard Code for Information
Interchange}, EBCDIC\footnote{Extended Binary Coded Decimal Interchange Code
(from IBM)}), multibyte characters( UTF-8, UTF-8, UTF-32\footnote{Universal
Character Set Transformation Format},  ) and characters in different
encodings(all tree character encodings are used for cyrillic symbols: 
Windows-1251, Code page 855, ISO/IEC 8859-5). Each RPC client and server should
agree about charater encoding they will use.

In addition to that, problems can occur with the representation of integers
(sign-and-magnitude method, one’s complement, two’s complement) and
real data types( floating-point, fixed-point, binary-coded decimal and single
precision, half precision, double precision). Some machines have different
endianness\footnote{The terms little-endian and big-endian refer to the way in which a word of data is stored into sequential bytes of
memory. The first byte of a sequence may store either the least significant byte of a word (little-endian) or the most
significant byte of a word (big-endian). Endianness refers to how bytes and
16-bit half-words map to 32-bit words in the system memory. \cite{arm_endian}}.
If two different machines, one little-endian and other big-endian, are
communicating to each other, they should use common endianness, otherwise they
will accept bytes in wrong order and data will be invalid.

There was description of primitive data types in RPC so far, but client and
server not always send primitive data types to each other. There could also be a
complex data structures, that contains several primitives like characters,
numbers or just raw bytes. Client and server should be aware of structure of
messages they send and receive. Usually these stuctures are specified by
Interface Definition Language(IDL).
IDL is a  is a specification language used to describe a software component's interface.
IDLs describe an interface in a language-independent way, enabling
communication between software components that do not share same language and
platform.
An interface is firstly specified in an IDL and then compiled into a
client stub and a server stub. RPC-based middleware systems
offer an IDL to support application development\cite{tanenbaum07}.

In most cases communication scheme is well known and some standard
message protocol is used. Previous example of RPC system, that was written using
Python programming language, used XML-RPC protocol. XML-RPC defines XML data
types that are used in RPC messages. There are several alternative common used
schemes, that provide similar functionality. They all could be divided into
two groups: platform and programming language dependent and systems that  
can be used in multiplatform and multilanguage environment. XML-RPC belongs to
second group, the similar techologies are JSON-RPC, SOAP and CORBA.
Actually, most RPC implementations in the first group do not require any special
harware. They are programming language dependent and may be used with
the assumption, that components in the system are written using that specific
language. Some examples are provided below:
\begin{itemize}
  \item Java Remote Method Invocation
  \item Pyro object-oriented form of RPC for Python.
  \item Windows Communication Foundation ( .NET framework)
  \item \ldots   
\end{itemize} 
Most of these programming languages have multiplatform implementations and RPC
system may be built using various hardware.
  
\subsubsection{RPC summary}
This section described one another posible way of \gls{IPC}. RPC is used in many
distributed systems in obvious or imlicit way. It provides mechanism for calling 
subroutines or procedures on another computer or system. 
Communication in RPC is based on message passing.
The Client sends to the server a message containing request for method call.
The Server sends to the client a message with procedure results.
Modern software tools provide ready RPC libraries and application programmer has no need to explicitly reinvent whole RPC system from scratch.
You can build various distributed systems using RPC.
 



\subsection{Data serialization}
\begin{quotation}
\textit{
Serialization is a process for converting a data structure or object into a format that
can be transmitted through a wire, or stored somewhere for later
use~\cite{json_vs_yaml}.
}
\end{quotation}
Previous sections described some possible implementations of Service Oriented
Architecture. These technologies use client-server communication and
send information between client and server, that need to be understood at both
destinations. No matter how this information is sent, using resource/object representation in case of REST or
request/response message in case of RPC, it needs to be converted to format that
can be decoded with the user of that information. Common transmission scenario
can look like:
\begin{enumerate}
  \item Client wants to send a some information to a server. It has some data in
  memory and that data is in application specific format(object structure, text,
  image, movie file).
  \item Client \textbf{packs} his information into a message and sends it to
  server using any possible transport channel(email, paper mail, homing pigeon, tcp
  socket, etc).
  \item Server receives this message, \textbf{unpacks} the message and gets a
  piece of information that client wanted to send.
  \item Server reads the information and decides what to do with it.
\end{enumerate}

Process of packing information is called \textbf{serialization} (also deflating
or marshalling) and process of unpacking is called \textbf{deserialization}
(inflating or unmarshalling). 

There are lots of different ways and formats that can
be used. Which method and format to choose depends on the requirements set up on
the object or data, and the use for the serialization (sending or storing). The choice
may also affect the size of the serialized data as well as serialization/deserialization
performance in terms of processing time and memory usage\cite{json_vs_yaml}.
Next section describes possible serialization solutions.

\subsubsection{Serialization technologies}
Serialization is supported by many programming languages, which provide tools
and libraries for data serialization to different formats. Article
\cite{wikipedia:comparison_of_data_serialization_formats}
provides a summary of well known data serialization formats.
Most of them could be divided into two groups: human-readable text based formats
and binary formats. Both groups have their own advantages and disadvantages.
\autoref{tbl:data_ser_formats} shortly describes them. 

\begin{table}[h]
	\centering	
	\begin{tabularx}{\textwidth}{|X|X|X|}
		\hline
		\textbf{Property} & 
		\textbf{Binary formats}  	& 
		\textbf{Human-readable formats}	
	    
	    \tabularnewline
		\hline
		Format examples &
		Protocol Buffers from Google \newline 
		{Apache Thrift(TBinaryProtocol)} \newline
		BSON used in MongoDB database \newline
		MessagePack(\url{http://msgpack.org})\newline
		and most of native serialization
		mechanisms in various programming languages(Java, Python, .NET framework,
		C++ BOOST serialization) &
		\gls{XML} \newline \gls{JSON} \newline \gls{YAML}
	
	    \tabularnewline
		\hline	  
		
		Advantages & 
		The two main reasons why binary formats are usually proposed
		are for \textbf{size} and \textbf{speed}. \newline
		Typically use fewer CPU cycles and requre less memory.
		Binary data is transformed as is, no need to encode data bytes( image, video,
		etc)
		Better for larger datasets \newline
		Random data access.
		

		
		&
		
		Do not have to write extra tools to debug the input and output; you can open
		the serialized output with a text editor to see if it looks right. \newline
		Self-descriptive and easily recoverable. \newline
		No need to use programming issues like sizeof and little-endian vs.
		big-endian. \newline
		Platform and programming language independent. \newline
		Broad support by tools and libraries \newline
		
		
		
		
		\tabularnewline
		\hline
		Disadvantages &
		Not verbose. Hard to debug. \newline
		Fixed data structures. Hard to extend. \newline
		Not self-descriptive( it is hard to undestand for human where actual data
		starts in the array of bytes, ), has no data description( metadata, layout
		of the data)\newline Require special software and  highly customized data
		access algorithms.
		Hard to recover data after software version change (remember different MS
		office formats)
				
		&
		
		Binary data needs to be converted to text form( Base64). \newline
		Additional processing overhead (CPU and memory consumption). \newline
		Lot of redundancy.
		Representing your data as text is not always easy/possible or
		reasonable(video/audio streams, large matrices with numbers)
				
				
		\tabularnewline
		\hline
	\end{tabularx} 
	\caption{Comparison of binary and human-readable serialization formats}
	\label{tbl:data_ser_formats}
\end{table}

Text-based nature makes human-readable format a suitable choice
for applications where humans are expected to see the data,
such as in document editing or where debugging information
is needed. Binary formats are better for high speed and low latency
applications.

\subsubsection{XML vs JSON}
Choosing the right serialization format mostly depends on your data and
application. But if there is no any constraint what protocol to use, text
protocols are more preferable. They give you advantages like: verbosity,
extensibility, portability.

Most popular human and machine readable serialization formats are XML and JSON.

\paragraph{XML} ~\\
\begin{quotation}
\textit{
XML is hugely important. Dr Charles Goldfarb, who was personally involved in its
invention, claims it to be “the holy grail of computing, solving the problem of universal data interchange between dissimilar systems.” It is also a handy format for everything from configuration files to data and documents of almost any type.
~\cite{xml_intro}}
\end{quotation} 

The fundamental design considerations of XML include
simplicity and human readability\cite{NurseitovPRI09}.
W3C\footnote{World Wide Web Consortium} specifies the design goals for XML
like~\cite{w3c_xml}:
\begin{enumerate}
  \item XML shall be straightforwardly usable over the Internet.
  \item XML shall support a wide variety of applications.
  \item XML shall be compatible with SGML\footnotemark.
  \item It shall be easy to write programs which process XML documents.
  \item The number of optional features in XML is to be kept to the absolute minimum, ideally zero.
  \item XML documents should be human-legible and reasonably clear.
  \item The XML design should be prepared quickly.
  \item The design of XML shall be formal and concise.
  \item XML documents shall be easy to create.
  \item Terseness in XML markup is of minimal importance.  
\end{enumerate}
\footnotetext{Standard Generalized Markup Language. SGML is a system for
defining markup languages. Authors mark up their documents by representing
structural, presentational, and semantic information alongside
content.~\cite{html_spec}}

The primary uses for XML are Remote Procedure Calls and object serialization for transfer of
data between applications. 

Simple data structure in XML looks like:

\begin{listing}[H]
\begin{minted}[frame=lines,
               framesep=2mm]{xml}
<person>
	<firstname>John</firstname>
	<surname>Smith</surname>
	<email>john.smith@example.com</email>
	<mobile>1234567890</mobile>
</person>
\end{minted}
\caption{XML structure decribing abstract person}
\label{lst:xml_person_example}
\end{listing}


\paragraph{JSON} ~\\
JSON (JavaScript Object Notation) is a lightweight data-interchange
format\cite{json_org}.It is easy for humans to read and write and also is
easy for machines to parse and generate. JSON is based on JavaScript
Programming Language and is directly supported inside JavaScript. It has library
bindings for popular programming languages.

JSON is built on two structures\cite{json_org}:
\begin{itemize}
  \item A collection of name/value pairs. In various languages, this is realized
  as an object, record, struct, dictionary, hash table, keyed list, or associative array.
  \item An ordered list of values. In most languages, this is realized as an
  array, vector, list, or sequence.
\end{itemize}

The same sturucture from XML section looks in JSON encoding like:
\begin{listing}[H]
\begin{minted}[frame=lines,
               framesep=2mm]{json}
{
	"firstname" : "John",
	"surname" : "Smith",
	"email" : "john.smith@example.com",
	"mobile" : 1234567890
}
\end{minted}
\caption{JSON structure decribing abstract person}
\label{lst:json_person_example}
\end{listing}

JSON value can be a string in double quotes, or a number, or true or false or
null, or an object or an array. These structures can be nested.

JSON specification is quite easy and it is  described using only one
page~\cite{json_org}. This page provides also a list of programming languages
and libraries that support JSON. 

TODO HERE comes comparison.    



\subsection{SOA and Embedded Systems}

Resent topics had a review of available tools and technologies for implementing a
\gls{SOA} system. Most of them are not directly portable to embedded systems.
Related publications \cite{5470528, dguinard-rest-vs-ws} claim that SOA
tools need to be adapted to a constrained hardware by using more lightweight approaches.
Common possibilities are: use of more resource friendly protocols, use of
existing service protocols in a contstrained manner( low request per second
ratio or smaller payload/packet size), use of special constrained protocols,
which are designed specially for interaction between small devices.

This section will describe two possibilities of implementing SOA on an embedded
device, that are based on different research
papers\cite{coap_survey,4221180}.
\nameref{sec:DPWS} section will introduce a
Web Services based device mmcounication framework. \nameref{sec:CoAP} section
will cover RESTful device interactions.
\subsubsection{Devices Profile for Web Services}
\label{sec:DPWS}
The Devices Profile for Web Services (DPWS) was developed to enable secure Web
service capabilities on resource-constrained devices\cite{ws4d_dpws}.
DPWS was mainly developed by Microsoft and some printer device manufacturers.
DPWS allows sending secure messages to and from Web services, discovering a Web service dynamically, describing a Web service, subscribing to, and receiving events from a Web service.


\begin{center}
 \begin{figure}[h]
	\centering
	\includegraphics[width=0.5\textwidth]{../images/background/dpws-stack.png}
	\caption{DPWS protocol stack \cite{ws4d_dpws} }
	\label{fig:dpws_protocol_stack}
 \end{figure}
\end{center}


\autoref{fig:dpws_protocol_stack} shows DPWS protocol stack. It is based on  
several Web Services specifications\cite{ws4d_dpws}:
\begin{itemize}
  \item WS-Addressing for advanced endpoint and message addressing
  \item WS-Policy for policy exchange
  \item WS-Security for managing security
  \item WS-Discovery and SOAP-over-UDP for device discovery
  \item WS-Transfer / WS- Metadataexchange for device and service description
  \item WS-Eventing for managing subscriptions for event channels   
\end{itemize}
    
Like Web Services, DPWS uses SOAP, WSDL, XML-Schema. 


DPWS has been ported to several target software
platforms (Linux, Microsoft's Windows and Windows CE,
ExpressLogic's ThreadX and Quadros Systems' Quadros).
It was tested on a processor board comprising a 44-MHz ARM7
TDMI and associated memory (but no cache memory), running
ThreadX. The static memory footprint of the device software including
the OS, the TCP/IP protocol stack and the DPWS software is
less than 500 KB, while the dynamic memory requirements are
below 100 KB.\cite{4221180}.    
Another research \cite{5470528} reports that system disk space
requirements are between 61 and 478 Kbytes. 



\subsubsection{Constrained Application Protocol and Constrained RESTful Environments}
\label{sec:CoAP}

Constrained Application Protocol (CoAP) is a software protocol is
used in very simple electronics devices that allows them to communicate interactively over the Internet.
It is particularly targeted for small low power sensors, switches,
valves and similar components that need to be controlled or supervised remotely, through standard Internet networks.

Lots of applications over the Interent use REST architecture.
   The Constrained RESTful Environments (CoRE) research in IETF organization
   aims to realize the REST architecture in a suitable form for the most
   constrained nodes (e.g.  8-bit microcontrollers with limited RAM and ROM) and
   networks (e.g.  6LoWPAN, IPv6 over Low power Wireless Personal Area
   Networks. RFC4944)~\cite{coap_spec}.

   One of the main goals of CoAP is to design a generic web protocol for
   the special requirements of this constrained environment, especially
   considering energy, building automation and other machine-to-machine
   (M2M) applications.  The purpose of CoAP is to implement a subset of REST
   coupled with HTTP, but optimized for M2M applications.  Although CoAP could
   be used instead of HTTP interfaces because of more compact protocol, it
   also offers features for M2M such as built-in discovery, multicast support and asynchronous message
   exchanges.
      
     CoAP has the following main features:
\begin{itemize}
  \item Constrained web protocol fulfilling M2M requirements.
  \item UDP [RFC0768] binding with optional reliability supporting unicast
      and multicast requests.
  \item Asynchronous message exchanges.
  \item Low header overhead and parsing complexity.
  \item URI and Content-type support.
  \item Simple proxy and caching capabilities.
  \item A stateless HTTP mapping, allowing proxies to be built providing
      access to CoAP resources via HTTP in a uniform way or for HTTP
      simple interfaces to be realized alternatively over CoAP.
  \item Security binding to Datagram Transport Layer Security (DTLS)
      [RFC6347]
\end{itemize}
   
CoAP messages  of two types: requests and responses. 
CoAP uses a short fixed-length binary header (4 bytes) that may be
followed by compact binary options and a payload.
CoAP is by default bound to UDP and optionally to DTLS, providing a high level
of communications security~\cite{wikipedia:coap}.

\subsubsection{Performance issues}

Web services implementations on embedded
devices have quite big overhead. Web service messages are at least 5 times
larger than conventional messages(messages using usual data stuctures, not XML),
messages take at least 2.5 to 2 times longer than conventional
messages, consume more 2 times more energy\cite{5470528}.
Therefore this technology can be only used on hardware with high computational and power
posibilities, not for deeply embedded devices like low-cost microcontrollers for wireless sensors and actuators. 

The second techology, CoAP and RESTful services requires less resources. Its
highly optimized implementation could be executed on a device with 8.5 kB of ROM
and 1.5 kB of RAM \cite{6076698}. It uses Contiki OS and TCP/IP
implementations. This architecture could be a great candidate for  a remote
embedded service.

CoAP was specially designed for constrained devices, while DPWS is a profile
(may be also called a port) of Web services techology to devices, with
complexity of all WS-* technologies. DPWS require more powerful hardware and is
not suitable for the devices where CoAP was designed to run.


% TODO DDDDDDDDDDDDDDDDDDDDDDDDDDD
% Service
% -
% Oriented Architecture can be implemented in different ways. General focus is on
% whatever architecture gets the job done. SOAP and REST have their strengths and weaknesses
% and will be highly suitable to some
% applications and positively terrible for others. The decision of
% which to use depends entirely on the circumstances of the application.






% Main constraints - resources of embedded system.
% Target STM32 device.
% Absence of ethernet.
% Existing implementations require more resources than STM32 or similar has.
% TODO another constraints of embedded systems.

% TODO put is somewhere
 % The benefit of SOA is
% to allow simultaneous use and easy mutual data exchange between programs of different vendors without additional programming or making changes to the services.
 

% DDDDDDDDDDDDDDDDDDDDDDDDDDD



%\input{background/embedded_security.tex}




\subsection{Final target system requirements}
% TODO list of final requirements here. This list is needed to be proved by the
% implementation

All previous sections introduced ideas how various devices can communicate to
each other. Each described technology has its own pros and cons. In general, we
need to choose a technology without any drawbacks or a technology, which is the
smallest evil chosen from list of suitable ones.


Web Services have these advantages:
\begin{itemize}
  \item Service description
  \item Service discovery
  \item Portability and platform independence( XML)
  \item Standardized protocols and message structures.
  \item Ability to transfer messages using other transport protocol than HTTP
\end{itemize}

The RESTful approach enables to model our domain objects as
consistent, hierarchical URLs with predictable operations for \gls{CRUD}(GET,
POST, PUT, DELETE). It is also based on HTTP that comes with standard error
codes, message types( see also \nameref{sec:rest_multiple_resource_types}) and
generally does most of the transport hard work, so we benefit from not
needing to maintain any user-developed protocol and using ready and well defined
technologies. One of the main concepts of REST is that RESTful services are
stateless and do not store session information. This reduces
resource consumption and makes client-server applications more loosely coupled
and scalable.

REST and Web services have different ideas. REST is based on resources and their
representation, while Web Services use messages to send data and call remote
methods between server and client. This technology is based on very simple idea
of \gls{RPC}. RPC has some essential benefits over REST and
WS-* technologies: it does not require traditional transport like HTTP, TCP,
Ethernet or Wi-fi. It can be implemented using any radio link, serial line or
any other suitable transport, that is able to deliver response and request
messages. Web Services in theory could also be transport independent, but most
WS-* tools assume the HTTP(and underlying technologies) as de facto standard
\footnote{SOAP protocol transport independence started from SOAP
1.2~\cite{soap_protocol_spec}}. Support of other underlying protocols for
SOAP(the main messaging protocol in WS-*) should be implemented separately and
most standard tools does not have this, they just use standard HTTP. WS-* has a
huge overhead in doing simple things like implementing small light
controlling service in your room. This is not right tool for that. Simple RPC
would be enough.

\autoref{tbl:service_ideas} defines system requirements and features for service prototype in this
work. These are ideas that were kept in mind while developing of embedded
service system was in progress.

\begin{longtabu} to \textwidth {|X|X|}
	
\hline 
\multicolumn{1}{|l|}{\textbf{Feature}} & 
\multicolumn{1}{l|}{\textbf{Description}} \\ 
\hline 
\endfirsthead

\multicolumn{2}{l}%
{{\bfseries \tablename\ \thetable{} -- continued from previous page}} \\
\hline 
\multicolumn{1}{|l|}{\textbf{Feature}} & 
\multicolumn{1}{l|}{\textbf{Description}} \\ 
\hline  
\endhead

\hline \multicolumn{3}{|r|}{{Continued on next page}} \\ \hline
\endfoot

% \hline \hline
\endlastfoot

		Transport independent solution
		&
		Company did not defined final communication yet.
		This may be one of these: UART, Bluetooth, RF, HTTP.
		In this work i implement only first prototype of such system and 
		requirements are about to be changed in future. 
		Therefore i need to implement a portable solution that could be quickly ported
		to another environment.
		
		\tabularnewline
		\hline
			Service description and service contract &
			Gives overview of all service capabilities. \newline 
			Provides an interface definition language. \newline
			Is a specification( sometimes may be only available one) to the server and
			server interface.
	 
	    \tabularnewline
		\hline	
	    RPC based communication &
	    Our prototype needs to control coffee machine and execute some requested
	    actions. If we had an Ethernet, the design decision would be REST(with all
	    its benefits and philosophy).
	    Otherwise RPC is the most suitable solution here. This system is message
	    oriented. Some standard RPC solution should be used or ported to the
	    embedded device.
	    
	    \tabularnewline
		\hline	
		Lightweight and verbose messages &
		Microcontroller has limited resources. We cannot use huge messages for really
		small request data~\footnotemark.
		Some text based protocol need to be used.
		Binary protocols are evil, it is real hard to understand and debug them.
		
	    \tabularnewline
		\hline
		Simple design &
		Simple is better than complex. \newline
		Not like WS-* \\
		\tabularnewline
		\hline
		Modular architecture &
		System should be divided into separate functional modules.
		It should be easy to change the internal implementation of these modules, without a need for global refactoring.		
		\tabularnewline
		\hline	  

	\caption{Ideas about embedded service}
	\label{tbl:service_ideas}
\end{longtabu}

\footnotetext{internet resource \url{http://www.simple-is-better.org/rpc/} contains a comparison of different message formats)}


