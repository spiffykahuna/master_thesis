\newpage
\section{Introduction}

%\subsection{Contributions}



% This is a test of acronyms  \gls{ASIC} \\
% 
% This is a test of code listings
% \begin{listing}[H]
% 	\inputminted[linenos=true,
% 	fistline=32,
% 	firstnumber=32,
% 	lastline=60]{java}{../source/CoffeeMachineViaBlueTooth.java}
% 	\caption{Example of a listing.}
% 	\label{lst:example}
% \end{listing}


% At first here will be lots of words about this cruel world and how it was
% changed in resent years.
% 
% Speech about reusable components.
% 
% Speech about small mobile devices that are everywhere.
% 
% 
% 
% Communication.
% 
% Interaction.





Computers are very essential in our life. Computer is an electronic device that
is used in almost every field. It is very accurate, fast and can accomplish
many tasks easily. In early days computers were only used by the government
and army to solve different high computational tasks. After invention of
low-cost microprocessors, computers became available to every person. Nowadays  
there are billions of personal computers and they are almost at every home.

Present day computers may be divided into two groups: very big and very small systems. In one group are mainly servers and
server farms, and in the other are mainly embedded systems. 
The gap between these groups becomes more wider, because of the availability of new small and low-power devices, which computational power raises constantly.
Lot of people prefer now to buy a tiny laptop instead of traditional workstations with a monitor and computer case under the table.
There is also a more smaller group of devices, that are implemented for a particular purpose - embedded computers.  
Every home has several examples of embedded computers.
Any appliance that has a digital clock, for instance, has a small embedded microcontroller that performs no other task than to display the clock.
Modern cars have embedded computers onboard that control such things as ignition timing and anti-lock brakes using input from a number of different sensors.


Today, there is very little or no communication between embedded devices and large servers in the web.
The problem is not only in the communication infrastructure, because the current communication technologies are able to provide different wired and/or wireless connections.
The problem is how we design and implement embedded systems. While we
try to keep big systems as open as possible (since it is their primary role), we tend to seclude and isolate embedded systems without providing easy ways to add a
custom interface to them. Embedded systems are still mainly seen as vendor-specific and task-oriented products, and not as components that can be easily manipulated and reused.

If all classes of devices could speak the same language, they could 
talk directly to each other in ways natural to the application without artificial technical barriers. This would allow easily 
creating seamless applications that aggregate the capabilities of all 
the electronics. The interoperation adds value to all the devices.

This idea comes from conception of Internet of Things (IoT). The Internet of
Things is the network of physical objects that can communicate to each other
using Internet and embedded technologies. This
connections compose an complex system where each member can send information
about its state and acquire data about other parties without any intervention of
human being.
For example sensors at your home could communicate to heater
and ventilation system and control temperature and humidity or your alarm can
tell all other devices that you are going to wake up soon. 
This technologies could help to track and count everything and improve the
quality of our lives by removing the unnecessary waste and additional cost.


The Internet of Things is quite popular topic of research nowadays. It possibly
can change the world like the Internet did. Many companies and universities are
trying to find and invent reasonable technologies for implementing this approach.

One of the methods how this communication can be performed is the concept of
remote services and service oriented architecture (SOA).
World Wide Web Consortium (W3C) defines a "Web service" as: 
\begin{quote}
A Web service is a software system designed to support interoperable machine-to-machine interaction over a network.
It has an interface described in a machine-processable format (specifically
\gls{WSDL}).
Other systems interact with the Web service in a manner prescribed by its
description using \gls{SOAP}-messages, typically conveyed using \gls{HTTP} with
an \gls{XML} serialization in conjunction with other Web-related standards.
\end{quote}

Services are unassociated, loosely coupled units of functionality.
Not only large server system are capable of providing this functionality.
Services can also be applicable in the resource-constrained embedded devices.

This work would introduce the concepts how \gls{SOA} can be in the  context  of 
embedded  systems.
This contains some research of already available technologies for
machine-to-machine communications and the implementation of small system
prototype, which contains two connected devices and uses service approach.

\subsection{Impact}
The impact of the research in this thesis has been started during the accomplishment of internship at the university. 
I was worked for some company and my task was to develop \gls{HMI} interface to some embedded system.
We were using wireless communication between the control unit and the machine it was controlling.
Control unit was a smartphone that was sending commands through Bluetooth protocol.
On the other side there was a coffee machine that was receiving and executing that commands.

At the same time i was studying how large enterprise systems communicate to each other.
I was reading about web services and related technologies.
Then was born an idea that there could also be a "small" device network, where
devices communicate to each other.

This was a research project and developed prototype could potentially become a real product.
In that case it needs to be connected to existing infrastructure.
Coffee machine could provide different remote services: remote coffee product
ordering, coffee machine maintenance and acquisition of statistical data, remote
payment.
This could look like traditional coffee automatic machines at the streets that
accept cash, but with a remote wireless control.

I stated to mine the information about different control possibilities.
This is how this research became a topic of my master thesis. 

\subsection{The task}

The purpose of this work is to create a prototype system which has remote
service capabilities and make a research of available machine-to-machine
communication possibilities.
This should be an universal and platform independent architecture, which can be easily ported to any suitable hardware and connected
into existing infrastructure. 

Already existing hardware are two STM32f103xx family microcontrollers which have
20 and 64 Kbytes of \gls{RAM}, 128 and 512 Kbytes of flash memory. They are also
equipped with \gls{UART} communication and whole communication need to work using
serial line. Remote server with service capabilities should work on that
hardware. General requirement for the hardware is low-cost microcontroller with
some connectivity, that does not make the  already existing system more
expensive and in the same time fulfil all required functionality.

Embedded server will be connected to a target device, which is actually a coffee
machine, and handle requests from clients by executing various functions on
target device.
Server should provide a functional interface to the client and know about
available functions inside coffee machine. That interface need to be verbose and
easy to connect. 

Client will be executed on mobile phone and will communicate with remote server
using Bluetooth wireless technology. It might be any mobile platform, but
the organization decided to try Google Android smartphones first. 
Android device need to have running program with a graphical user interface,
which is able connect to remote service and accomplish functional needs.

We need to be able to switch existing client hardware architecture and the to
choose various clients for this embedded service. It might be an Android mobile
application, regular desktop computer program or even web application.
It is good to have one common client code that will be used in these different
environments. We should write a client library, which have bindings to the
remote service and handles all the communication. It should provide a
convenient interface to a library client.

Coffee machine application is only the example of such architecture.
We need a real world problem and device to show all capabilities of such system.
Controlled device and the client application could vary. It might be a
remote light control at home or any data acquisition and control system at the
production plant. During this work such universal and extensible system will be
built.

To summarize and make our goal more complete we should make a list and follow
it:
\begin{enumerate}
  \item Make a research about available device-to-device connectivity
  techniques, software standards and communication protocols. Get current state
  of the art and find a suitable technology for the existing environment
  \item Get familiar with already existing hardware tools, setup programming
  environment and write some sample programs to test the capabilities.
  \item Implement service oriented architecture on a microcontroller hardware.
  \item Write some test functionality to prove chosen approach.
  \item Implement a client library software module.
  \item Create a test application that will use this client module inside the
  system and show how it can be applied. 
\end{enumerate}

This work will cover every step in this list and author will try to complete it. 

\subsection{Outline}

The first section will introduce available technologies of implementing
machine-to-machine communications. 
The main research is about the Internet technologies and concept of remote web services, this is because the Internet
is already an interconnected network with lots of machines are doing distributed
computing and interacting with each other. 
Lots of problems are already solved there and there are available different technologies and tools.
Although, all these already implemented features are not limited  only with the
Internet and related technologies like \gls{TCP}/IP and HTTP. 
Some essential features may be extracted from there and ported to resource-constrained devices.
The first section will also cover some connection and interaction possibilities that embedded systems have.

\nameref{sec:implementation} section covers embedded system prototype, that uses
concepts from different machine-to-machine communication technologies. It
contains description of architecture and software and hardware was used. 
Embedded service was implemented on a microcontroller device. 
\nameref{sec:embedded_service_impl} section covers the hardware and software parts of the system.

\nameref{sec:java_library} section describes the details of Java client stub library.
There is also implementation of a demo client application  in the section next to it. 
This is the last section here.


